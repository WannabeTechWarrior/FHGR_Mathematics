\documentclass[a4paper, landscape, 6pt]{article}

% --------------------------------------------------------------------------------------------------------------------------------------------------------------------------
\usepackage[top=1.7cm, bottom=0.1cm, left=0.5cm, right=0.5cm, headheight=1cm, footskip=0.3cm]{geometry}
\usepackage{fancyhdr} % Customizable headers and footers
\usepackage{multicol} % Allows for multiple columns
\usepackage[utf8]{inputenc} % Ensures UTF-8 encoding
\usepackage[ngerman]{babel} % Language support for German
\usepackage{amsmath, amsfonts} % Math packages for symbols and fonts
\usepackage{siunitx} % Write math units 
\usepackage{lmodern} % Enhances font quality
\usepackage{graphicx} % For including images
\usepackage[normalem]{ulem} % Provides underlining capabilities
\usepackage[dvipsnames, table]{xcolor} % Adds color options, including for tables
\usepackage{enumitem} % Customizable lists
\usepackage{mathabx} % Additional math symbols
\usepackage{colortbl} % Color options for tables
\usepackage{mathtools} % Additional tools for mathematical typesetting
\usepackage{relsize} % For adjusting fontsizes of equations
\usepackage{scalefnt} % For adjusting fontsizes
\usepackage{wallpaper} % Allows adding wallpaper backgrounds
\usepackage{changepage} % Enables adjustments to margins
\usepackage{tikz} % Graphics for drawing and illustration
\usetikzlibrary{tikzmark} % Use TikZ's method of remembering a position on a page
\usetikzlibrary{arrows.meta}  % For advanced arrows
\usetikzlibrary{matrix}
\usepackage{pgfplots} % For visualizing plots
\usepackage{tabularx} % Advanced table formatting
\usepackage[skins]{tcolorbox} % Colored boxes with advanced options
\usepackage{lipsum} % For generating dummy text
\usepackage{bbm} % Font package, especially for blackboard bold characters
\usepackage{multirow} % Combines rows in tables
\usepackage{letltxmacro} % Advanced command management
\usepackage{float} % Control float placement
\usepackage{amssymb} % Additional math symbols
\usepackage{algorithm} % Algorithm environment
\usepackage{algpseudocode} % Pseudocode environment
\usepackage{leftidx} % Left sub/superscripts
\usepackage{empheq} % Emphasized equations
\usepackage{textcomp} % Additional text symbols
\usepackage{calc} % Arithmetic operations within LaTeX lengths
\usepackage[makeroom]{cancel} % Strike through terms in equations
\usepackage{sidecap} % Places captions to the side of figures
\usepackage{physics} % Additional physics-related commands (e.g., Dirac notation)
\usepackage{makecell} % More advanced cell formatting in tables

% Custom header format using fancyhdr package
\pagestyle{fancy}
\fancyhf{} % Clears default header/footer
\lhead{\textbf{\subject} - \semester} % Left header
\rhead{\author} % Right header
\fancyfoot[C]{Page - \thepage} % Center the page number at the bottom


% Custom bullet symbol for itemized lists
\renewcommand\textbullet{\ensuremath{\bullet}}

% Command to create circled numbers for lists
\newcommand*\circled[1]{\tikz[baseline=(char.base)]{
        \node[shape=circle,draw,inner sep=1.2pt] (char) {#1};}}

% Increase row height in tables
\renewcommand{\arraystretch}{1.5}

\newtcbox{\mathbox}[1][]{
    nobeforeafter, 
    colframe=white!30!black, % Frame color
    colback=white!20, % Background color
    boxrule=0.5pt, % Border thickness
    arc=0mm, % Rounded corners
    before skip = 1mm,
    right = 1mm,
    left=1mm,
    top=0.5mm,
    bottom=0.5mm,
    center, % Center the content
    #1
}

% Fill the remaining space with blank space (in multicols environment)
\newcommand{\fillblank}{
    \vfill\null\columnbreak
}

% Set the default font family to sans-serif
\renewcommand{\familydefault}{\sfdefault}

% Custom command for creating a cheatsheet environment
\newcommand{\cheatsheet}[1]{
    \let\underbrace\LaTeXunderbrace % Fixes issue with underbrace in custom environments
    \let\overbrace\LaTeXoverbrace % Fixes issue with overbrace in custom environments
    \begin{document}
    \setlength{\columnseprule}{0.4pt} % Sets the rule width between columns
    \footnotesize
    \begin{multicols*}{\cols} % Starts a multicolumn environment
        \setlength{\abovedisplayskip}{\abovedisplayskip - 4mm} % Reduces space above displayed equations
        \setlength{\belowdisplayskip}{\belowdisplayskip - 3mm} % Reduces space below displayed equations
    #1
    \end{multicols*}
}

% Adjust the space between paragraphs
\setlength{\parskip}{0.1cm}

% Custom section formatting
\renewcommand{\section}[1]{
    \vskip 1pt
    \par\noindent\textbf{\normalsize#1} % Section title in bold, normal size
    \vskip -8pt
    \par\noindent\smash{\rule[-1.5pt]{2cm}{2pt}}\rule{\linewidth  - 2cm}{0.5pt} % Custom underline
    \par\vspace{-0.5mm}
}

% Custom subsection formatting
\renewcommand{\subsection}[2][]{
    \vskip 1pt
    \par\noindent\textbf{\small#2} \textbf{\small#1} % Subsection title in bold, small size
    \vskip -8pt
    \par\noindent\smash{\rule[-1.5pt]{1cm}{2pt}}\rule{\linewidth - 1cm}{0.5pt} % Custom underline
    \par\vspace{-1mm}
}

% Custom subsubsection formatting
\renewcommand{\subsubsection}[1]{
    \vskip 1pt
    \par\noindent\textbf{\footnotesize#1} % Subsubsection title in bold, footnote size
    \vskip -8pt
    \rule{\linewidth}{0.5pt} % Horizontal rule across the width
    \par\vspace{-1mm}
}

% No paragraph indentation
\setlength{\parindent}{0pt}
% Column settings for multicol environment
\setlength\columnsep{2mm}
\setlength{\columnseprule}{0pt}
\setlength{\fboxrule}{0.5pt}
\setlength{\headsep}{0.5em}
\setlength{\footskip}{12.0pt}

\def\subject{Mathematik}
\def\semester{2. Semester}
\def\author{Michael Graber, Joshua Kohler, Sven Gahlinger, Oliver Schütz}
\def\cols{3}

% Useful packages
\usepackage{amsmath}
\usepackage{graphicx}
\usepackage[colorlinks=true, allcolors=blue]{hyperref}
\usepackage{makecell}
\usepackage{adjustbox}
\usepackage{pgfplots}

\title{Mathe 2 Cheatsheet}

\cheatsheet{
    \section{Analysis}
    \subsection{Trigonometrie}
    \begin{tikzpicture}[scale=2.5, transform shape, cap=round,>=latex, every node/.style={scale=0.25}] 
    % draw the coordinates
    \draw[-] (-1.05cm,0cm) -- (1.05cm,0cm);
    \draw[-] (0cm,-1.05cm) -- (0cm,1.05cm);

    % draw the unit circle
    \draw[thick] (0cm,0cm) circle(1cm);

    \foreach \x in {0,30,45,60,90,120,135,150,180,210,225,240,270,300,315,330} {
            % lines from center to point
            \draw[gray] (0cm,0cm) -- (\x:1cm);
            % dots at each point
            \filldraw[black] (\x:1cm) circle(0.4pt);
            % draw each angle in degrees
            \draw (\x:0.6cm) node[fill=white] {$\x^\circ$};
    }

    \draw[Cerulean](0cm,0cm) -- (15:1cm);
    \filldraw[Cerulean] (15:1cm) circle(0.4pt);
    \draw (15:0.7cm) node[Cerulean,fill=white] {$\theta$};
    \draw (13:1cm) node[above right,Cerulean] {$\left(\cos(\theta), \sin(\theta)\right)$};

    % draw each angle in radians
    \foreach \x/\xtext in {
        30/\frac{\pi}{6},
        45/\frac{\pi}{4},
        60/\frac{\pi}{3},
        90/\frac{\pi}{2},
        120/\frac{2\pi}{3},
        135/\frac{3\pi}{4},
        150/\frac{5\pi}{6},
        180/\pi,
        210/\frac{7\pi}{6},
        225/\frac{5\pi}{4},
        240/\frac{4\pi}{3},
        270/\frac{3\pi}{2},
        300/\frac{5\pi}{3},
        315/\frac{7\pi}{4},
        330/\frac{11\pi}{6},
        360/2\pi}
            \draw (\x:0.85cm) node[fill=white] {$\xtext$};

    \node[right] at (0:1.05cm) {$\left(1,0\right)$};
    \node[above] at (90:1.05cm) {$\left(0,1\right)$};
    \node[left] at (180:1.05cm) {$\left(-1,0\right)$};
    \node[below] at (270:1.05cm) {$\left(0,-1\right)$};

    \foreach \x/\xtext/\y in {
        150/-\frac{\sqrt{3}}{2}/\frac{1}{2},
        135/-\frac{\sqrt{2}}{2}/\frac{\sqrt{2}}{2},
        120/-\frac{1}{2}/\frac{\sqrt{3}}{2}}
            \draw (\x:1cm) node[above left] {$\left(\xtext,\y\right)$};

    
    \foreach \x/\xtext/\y in {
        30/\frac{\sqrt{3}}{2}/\frac{1}{2},
        45/\frac{\sqrt{2}}{2}/\frac{\sqrt{2}}{2},
        60/\frac{1}{2}/\frac{\sqrt{3}}{2}}
            \draw (\x:1cm) node[above right] {$\left(\xtext,\y\right)$}; 
    
    \foreach \x/\xtext/\y in { 
        210/-\frac{\sqrt{3}}{2}/-\frac{1}{2},
        225/-\frac{\sqrt{2}}{2}/-\frac{\sqrt{2}}{2},
        240/-\frac{1}{2}/-\frac{\sqrt{3}}{2} }
            \draw (\x:1cm) node[below left] {$\left(\xtext,\y\right)$};

    \foreach \x/\xtext/\y in {
        330/\frac{\sqrt{3}}{2}/-\frac{1}{2},
        315/\frac{\sqrt{2}}{2}/-\frac{\sqrt{2}}{2},
        300/\frac{1}{2}/-\frac{\sqrt{3}}{2}}
            \draw (\x:1cm) node[below right] {$\left(\xtext,\y\right)$};
\end{tikzpicture}
    \usetikzlibrary{datavisualization.formats.functions,backgrounds,calc}
\def\mytypesetter#1{% page 813
  \pgfmathparse{#1/pi}%
  \pgfmathprintnumber{\pgfmathresult}$\pi$%
}

\begin{tikzpicture}[scale=1.4, transform shape, cap=round,>=latex, every node/.style={scale=0.25}] 
\draw[->] (-1.5cm,0cm) -- (1.5cm,0cm) node[right,fill=white] {$x$};
\draw[->] (0cm,-1.5cm) -- (0cm,1.5cm) node[above,fill=white] {$y$};

\draw[thick] (0cm,0cm) circle(1cm);

\foreach \x in {0,30,...,360} {
    \draw[gray] (0cm,0cm) -- (\x:1cm);
    \filldraw[black] (\x:1cm) coordinate (x\x) circle (0.4pt);
    \draw (\x:0.6cm) node[fill=white] {$\x^\circ$};
}

\foreach \x/\xtext in {
    30/\frac{\pi}{6},
    45/\frac{\pi}{4},
    60/\frac{\pi}{3},
    90/\frac{\pi}{2},
    120/\frac{2\pi}{3},
    135/\frac{3\pi}{4},
    150/\frac{5\pi}{6},
    180/\pi,
    210/\frac{7\pi}{6},
    225/\frac{5\pi}{4},
    240/\frac{4\pi}{3},
    270/\frac{3\pi}{2},
    300/\frac{5\pi}{3},
    315/\frac{7\pi}{4},
    330/\frac{11\pi}{6},
    360/2\pi}
\draw (\x:0.85cm) node[fill=white] {$\xtext$};

\foreach \x/\xtext/\y in {
    30/\frac{\sqrt{3}}{2}/\frac{1}{2},
    45/\frac{\sqrt{2}}{2}/\frac{\sqrt{2}}{2},
    60/\frac{1}{2}/\frac{\sqrt{3}}{2},
    150/-\frac{\sqrt{3}}{2}/\frac{1}{2},
    135/-\frac{\sqrt{2}}{2}/\frac{\sqrt{2}}{2},
    120/-\frac{1}{2}/\frac{\sqrt{3}}{2},
    210/-\frac{\sqrt{3}}{2}/-\frac{1}{2},
    225/-\frac{\sqrt{2}}{2}/-\frac{\sqrt{2}}{2},
    240/-\frac{1}{2}/-\frac{\sqrt{3}}{2},
    330/\frac{\sqrt{3}}{2}/-\frac{1}{2},
    315/\frac{\sqrt{2}}{2}/-\frac{\sqrt{2}}{2},
    300/\frac{1}{2}/-\frac{\sqrt{3}}{2}}
\draw (\x:1.25cm) node {$\left(\xtext,\y\right)$};

\draw (-1.25cm,0cm) node[above=1pt] {$(-1,0)$}
(1.25cm,0cm)  node[above=1pt] {$(1,0)$}
(0cm,-1.25cm) node[fill=white] {$(0,-1)$}
(0cm,1.25cm)  node[fill=white] {$(0,1)$};

\begin{scope}[xshift=20mm]
    \datavisualization
    [
    school book axes,
    y axis={unit length=10mm},
    x axis={unit length=2.5mm, ticks={step=(.5*pi), tick typesetter/.code=\mytypesetter{##1}}},
    visualize as smooth line,
    ]
    data [format=function] {
    var x : interval [0:4*pi];
    func y = sin(\value x r);
    };
\end{scope}

\begin{scope}[on background layer]
    \coordinate (o) at (0,0);
    \foreach \i in {90,120,...,270}
    {
        \draw [densely dashed, opacity=.85, color=blue!80!cyan] (x\i) -- ({x\i} -| o) -- ++(20mm,0) -- ++(pi*\i/720,0) coordinate (xx\i) edge (xx\i |- o)  -- ++(.5*pi,0) coordinate (xxx\i) edge (xxx\i |- o);
    }
\end{scope}
\end{tikzpicture}


    \begin{tabularx}{\linewidth}{|X|X|X|X|}
    \hline
    \textbf{Sin} & \textbf{Cos} & \textbf{Tan} & \textbf{Cot}\\
    \hline
    \textbf{G} & \textbf{A} & \textbf{G} & \textbf{A}\\
    \hline
    \textbf{H} & \textbf{H} & \textbf{A} & \textbf{G}\\
    \hline
\end{tabularx}

\begin{tikzpicture}
\begin{axis}[enlargelimits=false,
             axis lines=middle,
             scale=1.2,
             xtick={-3.15159, -1.57080, 0,
                     1.57080,  3.15159, 4.71239,
                     6.28318,  7.85398, 9.42478 }, 
             xticklabels={$-\pi$, $-\frac{1}{2}\pi$, 0,
                          $\frac{1}{2}\pi$, $\pi$, $\frac{3}{2}\pi$,
                          $2\pi$, $\frac{5}{2}\pi$, $3\pi$ },
             ytick={-3,-2,-1,0,1,2,3},
             grid=major, % only a grid on the defined ticks
             samples=100 % number of points
             ]
 
  % sin
  \addplot[blue,no marks,domain=-1.2*pi:3*pi]{sin(deg(x))}; % deg to convert radians
  \node[right=10pt,above] at (axis cs:5*pi/2,1){\color{blue}$\sin(x)$};
 
  % cos
  \addplot[red,no marks,domain=-1.2*pi:3*pi] {cos(deg(x))};
  \node[above left] at (axis cs:2*pi,1){\color{red}$\cos(x)$};
 
  % tan, multiple parts because of singularities
  \addplot[orange,no marks,domain=-1.2*pi:-0.583*pi, ]{tan(deg(x))};
  \addplot[orange,no marks,domain=-0.4*pi:5*pi/12,   ]{tan(deg(x))};
  \addplot[orange,no marks,domain=27*pi/45:17*pi/12, ]{tan(deg(x))};
  \addplot[orange,no marks,domain=1.6*pi:29*pi/12,   ]{tan(deg(x))};
  \addplot[orange,no marks,domain=2.6*pi:36*pi/12,   ]{tan(deg(x))};
  \node[right] at (axis cs:pi/2,2.5){\color{orange}$\tan(x)$};
 
\end{axis}
\end{tikzpicture}
    \subsection{Integrale}
    \subsubsection{Substitution}
\textbf{Normale Substitution}\\
\begin{align}
    \int_{a}^{b} f(x) dx \qquad \large|& \qquad u(x) = f(x)\notag\\
    |& \qquad u'(x) = f'(x)\notag\\
    |& \qquad du = u'(x)\cdot dx\notag\\
    \int_{u(a)}^{u(b)} u(x) \frac{1}{u'(x)} du \qquad \large|& \qquad dx = \frac{1}{u'(x)} \cdot du
\end{align}\\
\textbf{Umgekehrte Substitution}\\


    % https://en.wikipedia.org/wiki/List_of_integrals_of_trigonometric_functions#Integral_over_a_full_circle
\subsubsection{Standardintegrale}

\newcommand{\command}{\operatorname{new command output}}

\begin{flalign}
    &\text{Sinus}&\notag\\
    &\int{\sin{x} \cdot \cos{x}}\, dx = \frac{1}{2} \cdot \sin^2{x} + c&\\
    &\int{\sinh{x}} \,dx = \cosh{x} + c&\\
    &\text{Cosinus}&\notag\\
    &\int{\cosh{x}}\,dx = \sinh{x} + c&\\
    &\int{\cot{x}}\, dx = \frac{1}{\tan{x}} + c = \ln|\sin{x}| + c = \frac{\cos{x}}{\sin{x}} + c&\\
    &\int{\coth{x}}\, dx = \frac{\cosh{x}}{\sinh{x}} + c&\\
    &\text{Tangents}&\notag\\
    &\int{\tan{x}}\, dx = \frac{1}{\cot{x}} + c = -\ln|\cos{x}| + c = \frac{\sin{x}}{\cos{x}} + c&\\
    &\int{\tanh{x}}\, dx = \frac{\sinh{x}}{\cosh{x}} + c&
\end{flalign}

\textcolor{red}{Add these derrivatives somewhere usefull}
\begin{flalign}
    &\frac{d}{d{x}} \tan{x}\, = 1 + \tan^2{x}&\\
    &\frac{d}{d{x}} \cot{x} = -1 -\cot^2{x}&
\end{flalign}



    \subsubsection{Integralfläche berechnen (analytisch)}
Sobald sich Funktionen schneiden, muss das Integral aufgeteilt werden.
Wenn die Fläche über/unter der X-Achse berechnet werden soll, kann diese als Funktion $g(x) = 0$ angesehen werden.

\begin{flalign}
    A = \int_{a}^{b}{|f(x) - g(x)|} \,dx \label{eq:Calculate_area}
\end{flalign}

\textcolor{red}{Scale Plot and make a better example}
\begin{center}
\begin{tikzpicture}
    \begin{axis}[
        axis lines = middle,
        xlabel = \(x\),
        ylabel = {Function values},
        xmin = -0.5, xmax = 1.5,
        ymin = -0.5, ymax = 1.5,
        domain = 0:1,
        samples = 100,
        fill=blue!20,
        area style,
        legend style={at={(1,1)}, anchor=south east},
    ]
    \addplot[blue, thick] {x^2};
    \addplot[red, thick] {x};
    % \addplot[blue!20] fill between[of={x^2}, and={x}]; here is some compiling problem which I couldn't solve at the moment
    \legend{$f(x) = x^2$,$g(x) = x$}
    \end{axis}
\end{tikzpicture}
\end{center}

\textbf{Anleitung}
\begin{enumerate}
    \item Funktionen gleich setzen, um Nullstellen zu berechnen
    \item Integrale bilden
    \item Berechnen
\end{enumerate}

\subsubsection{Trapezformel (Numerisch)}
\begin{flalign}
    &S_{1} = y_{1} + y_{n} \qquad S_{2} = y_{1} + y_{2} + \dots + y_{n-1}&\notag\\
    &\frac{1}{2} \cdot h \cdot S_{1} + h \cdot S_{2}&\label{eq:Trapezformel}
\end{flalign}
\begin{enumerate}
    \item Stützstellen bestimmen und ausrechnen
    \item Mit Taschenrechner alle Stützstellen mittels Formel \ref{eq:Trapezformel} zusammenrechnen 
\end{enumerate}

    \subsubsection{Volumenintegral berechnen}
\begin{flalign}
    V = \pi \int_{a}^{b}{f(x)^{2}} \,dx \label{eq:Rotationsvolumen}
\end{flalign}

\textbf{Anleitung}
\begin{enumerate}
    \item Rotation um Y-Achse \rightarrow \, Umkehrfunktion bestimmen
    \item Mit der Formel \ref{eq:Rotationsvolumen} das Volumen berechnen
\end{enumerate}
    \section{Komplexe Zahlen}
    
\[
\arg(z) =
\begin{cases} 
    0                                                & \text{if } \Re(z) \geq 0 \land \Im(z) = 0 \quad | \quad \text{CASE 1}\\
    \arctan\left(\frac{\Im(z)}{\Re(z)}\right)        & \text{if } \Re(z) > 0 \land \Im(z) > 0 \quad | \quad \text{CASE 2}\\
    \frac{\pi}{2}                                    & \text{if } \Re(z) = 0 \land \Im(z) > 0 \quad | \quad \text{CASE 3}\\
    \arctan\left(\frac{\Im(z)}{\Re(z)}\right) + \pi  & \text{if } \Re(z) < 0 \qquad \qquad \qquad \quad| \quad \text{CASE 4}\\
    \pi                                              & \text{if } \Re(z) < 0 \land \Im(z) = 0 \quad | \quad \text{CASE 5}\\
    \frac{3\pi}{2}                                   & \text{if } \Re(z) = 0 \land \Im(z) < 0 \quad | \quad \text{CASE 6}\\
    \arctan\left(\frac{\Im(z)}{\Re(z)}\right) + 2\pi & \text{if } \Re(z) > 0 \land \Im(z) < 0 \quad | \quad \text{CASE 7}
\end{cases}
\]

\begin{tikzpicture}[scale=1]

% Axes
\draw[->] (-2,0) -- (2,0) node[right] {$\Re(z)$};
\draw[->] (0,-2) -- (0,2) node[above] {$\Im(z)$};

% Regions
% Case 1: Re(z) >= 0 and Im(z) = 0 (positive real axis)
\draw[thick, red] (0,0) -- (2,0) node[midway, below] {Case 1};

% Case 2: Re(z) > 0 and Im(z) > 0 (upper-right quadrant)
\fill[green, opacity=0.3] (0,0) -- (0,2) -- (2,2) -- (2,0) -- cycle;
\node[green!50!black] at (1.5,1.5) {Case 2};

%  Case 3: Re(z) = 0 and Im(z) > 0 (positive imaginary axis)
\draw[thick, cyan] (0,0) -- (0,2) node[midway, left] {Case 3};

\fill[purple, opacity=0.3] (0,0) -- (0,2) -- (-2,2) -- (-2,-2) -- (0,-2) -- cycle;
% Case 4: Re(z) < 0 and Im(z) > 0 (upper-left quadrant)
\node[purple!70!black] at (-1.5,1.5) {Case 4};
% Case 4: Re(z) < 0 and Im(z) < 0 (lower-left quadrant)
\node[purple!70!black] at (-1.5,-1.5) {Case 4};

% Case 5: Re(z) < 0 and Im(z) = 0 (negative real axis)
\draw[thick, blue] (0,0) -- (-2,0) node[midway, below] {Case 5};

% Case 6: Re(z) = 0 and Im(z) < 0 (negativ imaginary axis)
\draw[thick, magenta] (0,0) -- (0,-2) node[midway, left] {Case 6};

% Case 7: Re(z) > 0 and Im(z) < 0 (lower-right quadrant)
\node[yellow!20!black] at (1.5,-1.5) {Case 7};
\fill[yellow, opacity=0.3] (0,0) -- (0,-2) -- (2,-2) -- (2,0) -- cycle;

% Origin
\filldraw[black] (0,0) circle (1pt) node[below right] {$0$};

\end{tikzpicture}
    

\subsubsection{Koordinaten Arten}
\begin{flalign}
    &\textbf{Kartesische Koordinaten}& \notag \\
    &z = x + iy &\\[1ex]
    &\textbf{Polarkoordinaten} & \notag \\
    &\text{Umrechnung kartesisch} \to \text{polar:} & \notag \\
    &r = \sqrt{x^2 + y^2} & \notag \\
    &\varphi = \arctan\left(\frac{y}{x}\right) \quad (x \neq 0) & \notag \\[1ex]
    &\textbf{Komplexe Polarform} & \notag \\
    &\operatorname{cis} \varphi = \cos \varphi + i \sin \varphi & \notag \\
    &z = r \operatorname{cis} \varphi = r(\cos \varphi + i \sin \varphi) & \label{eq:polarform} \\[1ex]
    &\textbf{Euler-Form} & \notag \\
    &z = re^{i\varphi} \quad \text{(äquivalent zur Polarform)} & \label{eq:euler}
\end{flalign}

\begin{flalign}
    &x = r \cdot \cos{\phi} \quad|\quad y = r \cdot \sin{\phi}&\\
    &\varphi = \arctan{\frac{x}{y}}&\\
    &z = re^{i\varphi} = x + iy = r\cdot \operatorname{cis}{\varphi}&
\end{flalign}
    \subsubsection{Koordinaten Wechsel}

\begin{flalign}
    &\textbf{Kartesisch $\Rightarrow$ Polar}&\notag\\
    &
    &\textbf{Polar $\Rightarrow$ Kartesisch}&\notag\\
    &&\\
\end{flalign}
    \section{Lineare Algebra}
    \subsection{Vektoranalysis}
    \subsubsection{Begriffe Vektoren}
\begin{itemize}
    \item \textbf{Skalarprodukt}\\
    $\bullet \langle \vec{v}, \vec{w} \rangle = 0 \Rightarrow \vec{v} \perp \vec{w} \qquad \bullet \langle a \cdot \vec{v}, \vec{w} \rangle = a \cdot \langle \vec{v}, \vec{w} \rangle$\\
    $\bullet \left(\begin{matrix} x \\ y \end{matrix}\right) \perp \left(\begin{matrix} -y\\ x \end{matrix} \right) \qquad \bullet \angle(\vec{v}, \vec{w}) = \arccos(\frac{\langle \vec{v}, \vec{w}}{\left|\vec{v}\right| \cdot \left|\vec{w}\right|})$
\end{itemize}


\subsubsection{Kreuzprodukt}
\begin{minipage}{\linewidth}
    \begin{tikzpicture}[yscale=1.5, xscale=1.7, every node/.style={scale=1}]
    % Rechteck
    \draw[-,fill=white!95!red](0,0)--(3,0)--(4,1)--(1,1)--cycle;
    % Formel in der Fläche
    \node at (2,0.5) {$A = |\textcolor{violet}{\vec{c}} | = |\textcolor{blue}{\vec{a}}| \cdot |\textcolor{red}{\vec{b}}| \cdot \sin(\varphi)$};
    % a
    \draw[ultra thick,-latex,blue](0,0)--(3,0)node[midway,below]{$\vec{a}$};
    % b
    \draw[ultra thick,-latex,red](0,0)--(1,1)node[midway,above]{$\vec{b}$};
    % a x b
    \draw[ultra thick,-latex,blue!50!red](0,0)--(0,2)node[pos=0.7,right]{$\textcolor{blue}{\vec{a}} \times \textcolor{red}{\vec{b}} = \vec{c}$};
    \draw (0.6,0) arc [start angle=0,end angle=45,radius=0.6]
    node[pos=0.7,right]{$\theta$};
    \end{tikzpicture}
\end{minipage}
    \section{Variabletabbele}
    \section{Übungsaufgaben}
}


\end{document}