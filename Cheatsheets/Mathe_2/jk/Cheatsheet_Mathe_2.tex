\documentclass[a4paper, landscape, 6pt]{article}

% --------------------------------------------------------------------------------------------------------------------------------------------------------------------------
\usepackage[top=1.7cm, bottom=0.1cm, left=0.5cm, right=0.5cm, headheight=1cm, footskip=0.3cm]{geometry}
\usepackage{fancyhdr} % Customizable headers and footers
\usepackage{multicol} % Allows for multiple columns
\usepackage[utf8]{inputenc} % Ensures UTF-8 encoding
\usepackage[ngerman]{babel} % Language support for German
\usepackage{amsmath, amsfonts} % Math packages for symbols and fonts
\usepackage{siunitx} % Write math units 
\usepackage{lmodern} % Enhances font quality
\usepackage{graphicx} % For including images
\usepackage[normalem]{ulem} % Provides underlining capabilities
\usepackage[dvipsnames, table]{xcolor} % Adds color options, including for tables
\usepackage{enumitem} % Customizable lists
\usepackage{mathabx} % Additional math symbols
\usepackage{colortbl} % Color options for tables
\usepackage{mathtools} % Additional tools for mathematical typesetting
\usepackage{relsize} % For adjusting fontsizes of equations
\usepackage{scalefnt} % For adjusting fontsizes
\usepackage{wallpaper} % Allows adding wallpaper backgrounds
\usepackage{changepage} % Enables adjustments to margins
\usepackage{tikz} % Graphics for drawing and illustration
\usetikzlibrary{tikzmark} % Use TikZ's method of remembering a position on a page
\usetikzlibrary{arrows.meta}  % For advanced arrows
\usetikzlibrary{matrix}
\usepackage{pgfplots} % For visualizing plots
\usepackage{tabularx} % Advanced table formatting
\usepackage[skins]{tcolorbox} % Colored boxes with advanced options
\usepackage{lipsum} % For generating dummy text
\usepackage{bbm} % Font package, especially for blackboard bold characters
\usepackage{multirow} % Combines rows in tables
\usepackage{letltxmacro} % Advanced command management
\usepackage{float} % Control float placement
\usepackage{amssymb} % Additional math symbols
\usepackage{algorithm} % Algorithm environment
\usepackage{algpseudocode} % Pseudocode environment
\usepackage{leftidx} % Left sub/superscripts
\usepackage{empheq} % Emphasized equations
\usepackage{textcomp} % Additional text symbols
\usepackage{calc} % Arithmetic operations within LaTeX lengths
\usepackage[makeroom]{cancel} % Strike through terms in equations
\usepackage{sidecap} % Places captions to the side of figures
\usepackage{physics} % Additional physics-related commands (e.g., Dirac notation)
\usepackage{makecell} % More advanced cell formatting in tables

% Custom header format using fancyhdr package
\pagestyle{fancy}
\fancyhf{} % Clears default header/footer
\lhead{\textbf{\subject} - \semester} % Left header
\rhead{\author} % Right header
\fancyfoot[C]{Page - \thepage} % Center the page number at the bottom


% Custom bullet symbol for itemized lists
\renewcommand\textbullet{\ensuremath{\bullet}}

% Command to create circled numbers for lists
\newcommand*\circled[1]{\tikz[baseline=(char.base)]{
        \node[shape=circle,draw,inner sep=1.2pt] (char) {#1};}}

% Increase row height in tables
\renewcommand{\arraystretch}{1.5}

\newtcbox{\mathbox}[1][]{
    nobeforeafter, 
    colframe=white!30!black, % Frame color
    colback=white!20, % Background color
    boxrule=0.5pt, % Border thickness
    arc=0mm, % Rounded corners
    before skip = 1mm,
    right = 1mm,
    left=1mm,
    top=0.5mm,
    bottom=0.5mm,
    center, % Center the content
    #1
}

% Fill the remaining space with blank space (in multicols environment)
\newcommand{\fillblank}{
    \vfill\null\columnbreak
}

% Set the default font family to sans-serif
\renewcommand{\familydefault}{\sfdefault}

% Custom command for creating a cheatsheet environment
\newcommand{\cheatsheet}[1]{
    \let\underbrace\LaTeXunderbrace % Fixes issue with underbrace in custom environments
    \let\overbrace\LaTeXoverbrace % Fixes issue with overbrace in custom environments
    \begin{document}
    \setlength{\columnseprule}{0.4pt} % Sets the rule width between columns
    \footnotesize
    \begin{multicols*}{\cols} % Starts a multicolumn environment
        \setlength{\abovedisplayskip}{\abovedisplayskip - 4mm} % Reduces space above displayed equations
        \setlength{\belowdisplayskip}{\belowdisplayskip - 3mm} % Reduces space below displayed equations
    #1
    \end{multicols*}
}

% Adjust the space between paragraphs
\setlength{\parskip}{0.1cm}

% Custom section formatting
\renewcommand{\section}[1]{
    \vskip 1pt
    \par\noindent\textbf{\normalsize#1} % Section title in bold, normal size
    \vskip -8pt
    \par\noindent\smash{\rule[-1.5pt]{2cm}{2pt}}\rule{\linewidth  - 2cm}{0.5pt} % Custom underline
    \par\vspace{-0.5mm}
}

% Custom subsection formatting
\renewcommand{\subsection}[2][]{
    \vskip 1pt
    \par\noindent\textbf{\small#2} \textbf{\small#1} % Subsection title in bold, small size
    \vskip -8pt
    \par\noindent\smash{\rule[-1.5pt]{1cm}{2pt}}\rule{\linewidth - 1cm}{0.5pt} % Custom underline
    \par\vspace{-1mm}
}

% Custom subsubsection formatting
\renewcommand{\subsubsection}[1]{
    \vskip 1pt
    \par\noindent\textbf{\footnotesize#1} % Subsubsection title in bold, footnote size
    \vskip -8pt
    \rule{\linewidth}{0.5pt} % Horizontal rule across the width
    \par\vspace{-1mm}
}

% No paragraph indentation
\setlength{\parindent}{0pt}
% Column settings for multicol environment
\setlength\columnsep{2mm}
\setlength{\columnseprule}{0pt}
\setlength{\fboxrule}{0.5pt}
\setlength{\headsep}{0.5em}
\setlength{\footskip}{12.0pt}

\def\subject{Mathematik}
\def\semester{2. Semester}
\def\author{Michael Graber, Joshua Kohler, Sven Gahlinger, Oliver Schütz}
\def\cols{3}

% Useful packages
\usepackage{amsmath}
\usepackage{graphicx}
\usepackage[colorlinks=true, allcolors=blue]{hyperref}
\usepackage{makecell}
\usepackage{adjustbox}
\usepackage{pgfplots}

\title{Mathe 2 Cheatsheet}

\cheatsheet{
    \section{Allgemein}
    \subsubsection{Linearität}
\begin{flalign}
    &f(x) \cdot a = f(x \cdot a) \qquad \forall a \in \mathbf{R}, \forall x \in \mathbf{R}^{n}&\\
    &f(x) + f(y) = f(x + y) \qquad V_{x,y} \in \mathbf{R}^{n}&\\
    &\textbf{Lineares Beispiel: } f(x) = 3x \text{ mit } a=3,\ x=2,\ y=5&\notag\\
    &1.\ \text{Homogenitätstest: } f(2 \cdot 3) \stackrel{?}{=} 3 \cdot f(2) &\notag\\
    &\ \ \quad f(6) = 6 \cdot 3 = 18 \quad \text{vs.} \quad 3 \cdot 3 \cdot 2 = 3 \cdot 6 = 18 \quad \Rightarrow 18 = 18\ (\checkmark)&\notag\\
    &2.\ \text{Additivitätstest: } f(2) + f(5) \stackrel{?}{=} f(2+5) &\notag\\
    &\ \ \quad (3 \cdot 2) + (3 \cdot 5) = 6 + 15 = 21 \quad \text{vs.} \quad 3 \cdot 7 = 21 \quad \Rightarrow 21 = 21\ (\checkmark)&\notag\\
    &\textbf{Gegenbeispiel: } f(x) = x^2 \text{ mit } a=2,\ x=3,\ y=4&\notag\\
    &1.\ \text{Homogenitätstest: } f(3 \cdot 2) \stackrel{?}{=} 2 \cdot f(3) &\notag\\ 
    &\ \ \quad f(6) = 6^2 = 36 \quad vs. \quad 2 \cdot 3^2 = 18 \quad \Rightarrow 36 \ne 18\ (\times)&\notag\\
    &2.\ \text{Additivitätstest: } f(3) + f(4) \stackrel{?}{=} f(3+4) &\notag\\
    &\ \ \quad 3^2 + 4^2 = 9 + 16 = 25 \quad vs. \quad 7^2 = 49 \quad \Rightarrow 25 \ne 49\ (\times)&\notag
\end{flalign}


    \subsubsection{Injektiv}

\subsubsection{Surjektiv}

\subsubsection{Bijektiv}
    \section{Analysis}
    \subsubsection{Polynome}
    % Create variables
\newcommand{\varA}{\ensuremath{\textcolor{brown}{\frac{1}{2}}}} 
\newcommand{\varB}{\ensuremath{\textcolor{magenta}{\frac{3}{2}}}}
\newcommand{\TeilerPol}{{\color{blue}q(x)} }
\newcommand{\SolPol}{{\color{LimeGreen}S(x)} }
\newcommand{\RestPol}{{\color{purple}r(x)} }
\newcommand{\Pol}{{\color{teal}P(x)} }
\newcommand{\M}[2]{\tikzmarknode{#1}{#2}}

\subsubsection{Hornerschema}
\begin{minipage}{0.45\linewidth}
    \textbf{Benötigt/Wichtig}\\
    \begin{itemize}
        \item Leitkoeffizient von \TeilerPol = 1
        \item Erratene Nullstelle
        \item grad(r) < grad(q)
    \end{itemize}
    \textbf{Formel}\\
    \begin{flalign}
        &\frac{\Pol}{\TeilerPol} = \SolPol + \frac{\RestPol}{\TeilerPol}& \label{eq:Hornerschema}
    \end{flalign}
\end{minipage}
\hfill
\begin{minipage}{0.55\linewidth}
    \textbf{Möglichkeiten}
    \begin{itemize}
        \item Polynom an einer Steller auswerten
        \item Polynomdivision
        \item Zahlensystem umformen
        \item Taylor Reihen entwickeln \textcolor{red}{noch hinzufügen} % Mathepeter video: Polynome nach Potenzen entwickeln (mit Horner Schema)
    \end{itemize}
\end{minipage}

\textbf{Beispiel}\\
$P(x) = -2x^5 - 3x^4 - 8x^3 - 11x^2 + 4$\\

$q(x) = -2x^2 + 3x -1 = \boxed{\mathbf{\textcolor{cyan}{1}}} -2 (x^2 - \varB + \varA)$\\

\begin{tabular}{>{$}r<{$} | *{6}{>{$}r<{$} }}
    \boxed{\textcolor{cyan}{2}}  & \M{01}{-2} & \M{02}{-3} & \M{03}{-8} & \M{04}{-11} & \M{05}{0} & \M{06}{4}\\ 
        \hline
    \M{10}{\varA} & \M{11}{\cdot} & \M{12}{\cdot} & \M{13}{1} & \M{14}{3} & \M{15}{8} & \M{16}{16}\\
    \M{20}{\varB} & \M{21}{\cdot} & \M{22}{-3} & \M{23}{-9} & \M{24}{-24} & \M{25}{-48} & \cdot \\
        \hline
    \boxed{\textcolor{cyan}{4}}   & \multicolumn{4}{c}{{\color{LimeGreen}\boxed{\M{31}{-2} \quad \M{32}{-6} \quad \M{33}{-16} \quad \M{34}{-32}}}} & \multicolumn{2}{c}{{\color{purple}\boxed{\M{35}{-40} \quad \M{36}{20}}}} 
\end{tabular}

\begin{flalign*}
    &\frac{\Pol}{\TeilerPol} \Rightarrow \overbrace{-\frac12}^{\boxed{\mathbf{\textcolor{cyan}{5}}}} \left({\color{LimeGreen}\boxed{-2x^3 -6x^2 - 16x - 32}} + \frac{{\color{purple}\boxed{-40x + 20}}}{{\color{blue}\boxed{x^2 - \frac32 x + \frac12}}}\right)&\\
    &= x^3 + 3x^2 + 8x + 16 + \frac{-40x + 20}{-2x^2 + 3x -1}&
\end{flalign*}

\textbf{Anleitung}
\begin{enumerate}[label={\color{cyan}\arabic*.}]
    \item Leitkoeffizient von \TeilerPol auf 1 setzen
    \item Tabelle mit den Koeffizienten und den Punkten erstellen
    \item Den Pfeilen nach unten entlang \textcolor{red}{addieren} und nach oben  mit jeweiligem Zeilenfaktor (links) \textcolor{blue}{multiplizieren}
    \item Gr(\Pol) - Gr(\TeilerPol) = Grad(\SolPol). Der Rest gehört zu \RestPol
    \item Wenn Schritt 1 dann muss durch den ausgeklammerten Faktor geteilt werden. Bei \RestPol Faktor in den Nenner ziehen.
\end{enumerate}

\begin{tikzpicture}[overlay, remember picture]
    \draw[->, red, thin] (01) to node[red, left] {+} (31);
    \draw[->, blue, thin] (31) to node[sloped] {\cdot \varB} (22);
    \draw[->, blue, thin] (31) to [out=5, in=220] node[pos=0.8, above, sloped] {\cdot \varA} (13);
    \draw[->, red, thin] (02) to (32);
    \draw[->, blue, thin] (32) to (14);
    \draw[->, red, thin] (03) to [bend left] (33);
    \draw[->, blue, thin] (33) to (15);
    \draw[->, red, thin] (04) to [bend left] (34);
    \draw[->, blue, thin] (34) to (16);
    \draw[->, red, thin] (05) to [bend left] (35);
    \draw[->, red, thin] (06) to [bend left] (36);
\end{tikzpicture}

% Erase variables
\makeatletter
    \let\varA\@undefined 
    \let\varB\@undefined 
    \let\TeilerPol\@undefined
    \let\Pol\@undefined
    \let\RestPol\@undefined
    \let\SolPol\@undefined
    \let\M\@undefined
\makeatother
    \subsubsection{Nusstellen erraten}

    \subsection{Trigonometrie}
    \begin{tikzpicture}[scale=2.5, transform shape, cap=round,>=latex, every node/.style={scale=0.25}] 
    % draw the coordinates
    \draw[-] (-1.05cm,0cm) -- (1.05cm,0cm);
    \draw[-] (0cm,-1.05cm) -- (0cm,1.05cm);

    % draw the unit circle
    \draw[thick] (0cm,0cm) circle(1cm);

    \foreach \x in {0,30,45,60,90,120,135,150,180,210,225,240,270,300,315,330} {
            % lines from center to point
            \draw[gray] (0cm,0cm) -- (\x:1cm);
            % dots at each point
            \filldraw[black] (\x:1cm) circle(0.4pt);
            % draw each angle in degrees
            \draw (\x:0.6cm) node[fill=white] {$\x^\circ$};
    }

    \draw[Cerulean](0cm,0cm) -- (15:1cm);
    \filldraw[Cerulean] (15:1cm) circle(0.4pt);
    \draw (15:0.7cm) node[Cerulean,fill=white] {$\theta$};
    \draw (13:1cm) node[above right,Cerulean] {$\left(\cos(\theta), \sin(\theta)\right)$};

    % draw each angle in radians
    \foreach \x/\xtext in {
        30/\frac{\pi}{6},
        45/\frac{\pi}{4},
        60/\frac{\pi}{3},
        90/\frac{\pi}{2},
        120/\frac{2\pi}{3},
        135/\frac{3\pi}{4},
        150/\frac{5\pi}{6},
        180/\pi,
        210/\frac{7\pi}{6},
        225/\frac{5\pi}{4},
        240/\frac{4\pi}{3},
        270/\frac{3\pi}{2},
        300/\frac{5\pi}{3},
        315/\frac{7\pi}{4},
        330/\frac{11\pi}{6},
        360/2\pi}
            \draw (\x:0.85cm) node[fill=white] {$\xtext$};

    \node[right] at (0:1.05cm) {$\left(1,0\right)$};
    \node[above] at (90:1.05cm) {$\left(0,1\right)$};
    \node[left] at (180:1.05cm) {$\left(-1,0\right)$};
    \node[below] at (270:1.05cm) {$\left(0,-1\right)$};

    \foreach \x/\xtext/\y in {
        150/-\frac{\sqrt{3}}{2}/\frac{1}{2},
        135/-\frac{\sqrt{2}}{2}/\frac{\sqrt{2}}{2},
        120/-\frac{1}{2}/\frac{\sqrt{3}}{2}}
            \draw (\x:1cm) node[above left] {$\left(\xtext,\y\right)$};

    
    \foreach \x/\xtext/\y in {
        30/\frac{\sqrt{3}}{2}/\frac{1}{2},
        45/\frac{\sqrt{2}}{2}/\frac{\sqrt{2}}{2},
        60/\frac{1}{2}/\frac{\sqrt{3}}{2}}
            \draw (\x:1cm) node[above right] {$\left(\xtext,\y\right)$}; 
    
    \foreach \x/\xtext/\y in { 
        210/-\frac{\sqrt{3}}{2}/-\frac{1}{2},
        225/-\frac{\sqrt{2}}{2}/-\frac{\sqrt{2}}{2},
        240/-\frac{1}{2}/-\frac{\sqrt{3}}{2} }
            \draw (\x:1cm) node[below left] {$\left(\xtext,\y\right)$};

    \foreach \x/\xtext/\y in {
        330/\frac{\sqrt{3}}{2}/-\frac{1}{2},
        315/\frac{\sqrt{2}}{2}/-\frac{\sqrt{2}}{2},
        300/\frac{1}{2}/-\frac{\sqrt{3}}{2}}
            \draw (\x:1cm) node[below right] {$\left(\xtext,\y\right)$};
\end{tikzpicture}
    \usetikzlibrary{datavisualization.formats.functions,backgrounds,calc}
\def\mytypesetter#1{% page 813
  \pgfmathparse{#1/pi}%
  \pgfmathprintnumber{\pgfmathresult}$\pi$%
}

\begin{tikzpicture}[scale=1.4, transform shape, cap=round,>=latex, every node/.style={scale=0.25}] 
\draw[->] (-1.5cm,0cm) -- (1.5cm,0cm) node[right,fill=white] {$x$};
\draw[->] (0cm,-1.5cm) -- (0cm,1.5cm) node[above,fill=white] {$y$};

\draw[thick] (0cm,0cm) circle(1cm);

\foreach \x in {0,30,...,360} {
    \draw[gray] (0cm,0cm) -- (\x:1cm);
    \filldraw[black] (\x:1cm) coordinate (x\x) circle (0.4pt);
    \draw (\x:0.6cm) node[fill=white] {$\x^\circ$};
}

\foreach \x/\xtext in {
    30/\frac{\pi}{6},
    45/\frac{\pi}{4},
    60/\frac{\pi}{3},
    90/\frac{\pi}{2},
    120/\frac{2\pi}{3},
    135/\frac{3\pi}{4},
    150/\frac{5\pi}{6},
    180/\pi,
    210/\frac{7\pi}{6},
    225/\frac{5\pi}{4},
    240/\frac{4\pi}{3},
    270/\frac{3\pi}{2},
    300/\frac{5\pi}{3},
    315/\frac{7\pi}{4},
    330/\frac{11\pi}{6},
    360/2\pi}
\draw (\x:0.85cm) node[fill=white] {$\xtext$};

\foreach \x/\xtext/\y in {
    30/\frac{\sqrt{3}}{2}/\frac{1}{2},
    45/\frac{\sqrt{2}}{2}/\frac{\sqrt{2}}{2},
    60/\frac{1}{2}/\frac{\sqrt{3}}{2},
    150/-\frac{\sqrt{3}}{2}/\frac{1}{2},
    135/-\frac{\sqrt{2}}{2}/\frac{\sqrt{2}}{2},
    120/-\frac{1}{2}/\frac{\sqrt{3}}{2},
    210/-\frac{\sqrt{3}}{2}/-\frac{1}{2},
    225/-\frac{\sqrt{2}}{2}/-\frac{\sqrt{2}}{2},
    240/-\frac{1}{2}/-\frac{\sqrt{3}}{2},
    330/\frac{\sqrt{3}}{2}/-\frac{1}{2},
    315/\frac{\sqrt{2}}{2}/-\frac{\sqrt{2}}{2},
    300/\frac{1}{2}/-\frac{\sqrt{3}}{2}}
\draw (\x:1.25cm) node {$\left(\xtext,\y\right)$};

\draw (-1.25cm,0cm) node[above=1pt] {$(-1,0)$}
(1.25cm,0cm)  node[above=1pt] {$(1,0)$}
(0cm,-1.25cm) node[fill=white] {$(0,-1)$}
(0cm,1.25cm)  node[fill=white] {$(0,1)$};

\begin{scope}[xshift=20mm]
    \datavisualization
    [
    school book axes,
    y axis={unit length=10mm},
    x axis={unit length=2.5mm, ticks={step=(.5*pi), tick typesetter/.code=\mytypesetter{##1}}},
    visualize as smooth line,
    ]
    data [format=function] {
    var x : interval [0:4*pi];
    func y = sin(\value x r);
    };
\end{scope}

\begin{scope}[on background layer]
    \coordinate (o) at (0,0);
    \foreach \i in {90,120,...,270}
    {
        \draw [densely dashed, opacity=.85, color=blue!80!cyan] (x\i) -- ({x\i} -| o) -- ++(20mm,0) -- ++(pi*\i/720,0) coordinate (xx\i) edge (xx\i |- o)  -- ++(.5*pi,0) coordinate (xxx\i) edge (xxx\i |- o);
    }
\end{scope}
\end{tikzpicture}


    \begin{tabularx}{\linewidth}{|X|X|X|X|}
    \hline
    \textbf{Sin} & \textbf{Cos} & \textbf{Tan} & \textbf{Cot}\\
    \hline
    \textbf{G} & \textbf{A} & \textbf{G} & \textbf{A}\\
    \hline
    \textbf{H} & \textbf{H} & \textbf{A} & \textbf{G}\\
    \hline
\end{tabularx}

\begin{tikzpicture}
\begin{axis}[enlargelimits=false,
             axis lines=middle,
             scale=1.2,
             xtick={-3.15159, -1.57080, 0,
                     1.57080,  3.15159, 4.71239,
                     6.28318,  7.85398, 9.42478 }, 
             xticklabels={$-\pi$, $-\frac{1}{2}\pi$, 0,
                          $\frac{1}{2}\pi$, $\pi$, $\frac{3}{2}\pi$,
                          $2\pi$, $\frac{5}{2}\pi$, $3\pi$ },
             ytick={-3,-2,-1,0,1,2,3},
             grid=major, % only a grid on the defined ticks
             samples=100 % number of points
             ]
 
  % sin
  \addplot[blue,no marks,domain=-1.2*pi:3*pi]{sin(deg(x))}; % deg to convert radians
  \node[right=10pt,above] at (axis cs:5*pi/2,1){\color{blue}$\sin(x)$};
 
  % cos
  \addplot[red,no marks,domain=-1.2*pi:3*pi] {cos(deg(x))};
  \node[above left] at (axis cs:2*pi,1){\color{red}$\cos(x)$};
 
  % tan, multiple parts because of singularities
  \addplot[orange,no marks,domain=-1.2*pi:-0.583*pi, ]{tan(deg(x))};
  \addplot[orange,no marks,domain=-0.4*pi:5*pi/12,   ]{tan(deg(x))};
  \addplot[orange,no marks,domain=27*pi/45:17*pi/12, ]{tan(deg(x))};
  \addplot[orange,no marks,domain=1.6*pi:29*pi/12,   ]{tan(deg(x))};
  \addplot[orange,no marks,domain=2.6*pi:36*pi/12,   ]{tan(deg(x))};
  \node[right] at (axis cs:pi/2,2.5){\color{orange}$\tan(x)$};
 
\end{axis}
\end{tikzpicture}
    \subsection{Integrale}
    Satz von Fubini
    \subsubsection{Substitution}
\textbf{Normale Substitution}\\
\begin{align}
    \int_{a}^{b} f(x) dx \qquad \large|& \qquad u(x) = f(x)\notag\\
    |& \qquad u'(x) = f'(x)\notag\\
    |& \qquad du = u'(x)\cdot dx\notag\\
    \int_{u(a)}^{u(b)} u(x) \frac{1}{u'(x)} du \qquad \large|& \qquad dx = \frac{1}{u'(x)} \cdot du
\end{align}\\
\textbf{Umgekehrte Substitution}\\


    % https://en.wikipedia.org/wiki/List_of_integrals_of_trigonometric_functions#Integral_over_a_full_circle
\subsubsection{Standardintegrale}

\newcommand{\command}{\operatorname{new command output}}

\begin{flalign}
    &\text{Sinus}&\notag\\
    &\int{\sin{x} \cdot \cos{x}}\, dx = \frac{1}{2} \cdot \sin^2{x} + c&\\
    &\int{\sinh{x}} \,dx = \cosh{x} + c&\\
    &\text{Cosinus}&\notag\\
    &\int{\cosh{x}}\,dx = \sinh{x} + c&\\
    &\int{\cot{x}}\, dx = \frac{1}{\tan{x}} + c = \ln|\sin{x}| + c = \frac{\cos{x}}{\sin{x}} + c&\\
    &\int{\coth{x}}\, dx = \frac{\cosh{x}}{\sinh{x}} + c&\\
    &\text{Tangents}&\notag\\
    &\int{\tan{x}}\, dx = \frac{1}{\cot{x}} + c = -\ln|\cos{x}| + c = \frac{\sin{x}}{\cos{x}} + c&\\
    &\int{\tanh{x}}\, dx = \frac{\sinh{x}}{\cosh{x}} + c&
\end{flalign}

\textcolor{red}{Add these derrivatives somewhere usefull}
\begin{flalign}
    &\frac{d}{d{x}} \tan{x}\, = 1 + \tan^2{x}&\\
    &\frac{d}{d{x}} \cot{x} = -1 -\cot^2{x}&
\end{flalign}



    \subsubsection{Integralfläche berechnen (analytisch)}
Sobald sich Funktionen schneiden, muss das Integral aufgeteilt werden.
Wenn die Fläche über/unter der X-Achse berechnet werden soll, kann diese als Funktion $g(x) = 0$ angesehen werden.

\begin{flalign}
    A = \int_{a}^{b}{|f(x) - g(x)|} \,dx \label{eq:Calculate_area}
\end{flalign}

\textcolor{red}{Scale Plot and make a better example}
\begin{center}
\begin{tikzpicture}
    \begin{axis}[
        axis lines = middle,
        xlabel = \(x\),
        ylabel = {Function values},
        xmin = -0.5, xmax = 1.5,
        ymin = -0.5, ymax = 1.5,
        domain = 0:1,
        samples = 100,
        fill=blue!20,
        area style,
        legend style={at={(1,1)}, anchor=south east},
    ]
    \addplot[blue, thick] {x^2};
    \addplot[red, thick] {x};
    % \addplot[blue!20] fill between[of={x^2}, and={x}]; here is some compiling problem which I couldn't solve at the moment
    \legend{$f(x) = x^2$,$g(x) = x$}
    \end{axis}
\end{tikzpicture}
\end{center}

\textbf{Anleitung}
\begin{enumerate}
    \item Funktionen gleich setzen, um Nullstellen zu berechnen
    \item Integrale bilden
    \item Berechnen
\end{enumerate}

\subsubsection{Trapezformel (Numerisch)}
\begin{flalign}
    &S_{1} = y_{1} + y_{n} \qquad S_{2} = y_{1} + y_{2} + \dots + y_{n-1}&\notag\\
    &\frac{1}{2} \cdot h \cdot S_{1} + h \cdot S_{2}&\label{eq:Trapezformel}
\end{flalign}
\begin{enumerate}
    \item Stützstellen bestimmen und ausrechnen
    \item Mit Taschenrechner alle Stützstellen mittels Formel \ref{eq:Trapezformel} zusammenrechnen 
\end{enumerate}

    \subsubsection{Partialle Integration}
\textbf{Formel}\\
\begin{flalign}
    &\int f(x) \cdot g(x) \,dx&\notag\\
    &\Rightarrow \sum_{k=0}^{n-1}{(-1)^{k} \cdot f^{k}(x) \cdot g^{-1-k}(x) + (-1)^n \int f^{n}(x) \cdot g^{-n}(x) \,dx}&\\
    &\int{f(x) \cdot g(x)} \,dx = F(x) \cdot g(x) - \int{F(x) \cdot g(x)} \,dx &
\end{flalign}
\begin{tabular}{r|cc}
    Vorzeichen & Differenzieren & Integrieren\\
    \hline
    + & $\tikzmarknode{D0}{f}$ & g\\
    - & $\tikzmarknode{D1}{f^{1}}$ & $\tikzmarknode{I1}{g^{-1}}$\\
    + & $\tikzmarknode{D2}{f^{2}}$ & $\tikzmarknode{I2}{g^{-2}}$\\
    $\pm$ & $\tikzmarknode{D3}{\vdots}$ & $\tikzmarknode{I3}{\vdots}$\\
    $(-1)^{n-1}$ & $\tikzmarknode{D4}{f^{n-1}}$ & $\tikzmarknode{I4}{g^{-n+1}}$\\
    $(-1)^{n}$ & $\tikzmarknode{D5}{f^{n}}$ & $\tikzmarknode{I5}{g^{-n}}$\\
\end{tabular}

\begin{tikzpicture}[overlay, remember picture]
    \draw[->] (D0) to (I1);
    \draw[->] (D1) to (I2);
    \draw[->] (D2) to (I3);
    \draw[->] (D3) to (I4);
    \draw[->] (D4) to (I5);
    \draw[->] (D5) to (I5);
\end{tikzpicture}\\

\textbf{Beispiel}\\
\begin{flalign*}
    &\sqrt{2}\int_{0}^{2\pi}{\cos^2{x} + x \cdot \sin{x}} \,dx&
\end{flalign*}
\begin{minipage}{0.5\linewidth}
\begin{tabular}{r|cc}
    V & D & I\\
    \hline
    + & $\cos{x}$ & $\cos{x}$\\
    - & $-\sin{x}$ & $\sin{x}$\\
\end{tabular}\\

\begin{flalign*}
    &= \cos{x} \cdot \sin{x} + \int_{0}^{2\pi}{\sin^2{x}} \,dx&\\
    &F = \cos{x} \cdot \sin{x} + \int_{0}^{2\pi}{1} \,dx&\\ &- \int_{0}^{2\pi}{\cos^2{x}} \,dx&\\
    &= \frac{1}{2}(\cos{x} \cdot \sin{x} + x)&
\end{flalign*}
\end{minipage}
\hfill
\begin{minipage}{0.5\linewidth}
\begin{tabular}{r|cc}
    V & D & I\\
    \hline
    + & $x$ & $\sin{x}$\\
    - & $1$ & $-\cos{x}$\\
    + & $0$ & $-\sin{x}$
\end{tabular}\\

\begin{flalign*}
    &= x \cdot -\cos{x} + \sin{x}&\\ &- \int_{0}^{2\pi}{0 \cdot \sin{x}} \,dx&\\
    &= x \cdot -\cos{x} + \sin{x}&\\
\end{flalign*}
\end{minipage}\\

\begin{flalign*}
    &\Rightarrow \sqrt{2} \left[\frac{1}{2} (\cos{x} \cdot \sin{x} + x + \sin{x} - x \cdot \cos{x})\right]_{0}^{2\pi}&\\
\end{flalign*}

    \subsubsection{Volumenintegral berechnen}
\begin{flalign}
    V = \pi \int_{a}^{b}{f(x)^{2}} \,dx \label{eq:Rotationsvolumen}
\end{flalign}

\textbf{Anleitung}
\begin{enumerate}
    \item Rotation um Y-Achse \rightarrow \, Umkehrfunktion bestimmen
    \item Mit der Formel \ref{eq:Rotationsvolumen} das Volumen berechnen
\end{enumerate}
    \section{Komplexe Zahlen}
    \subsubsection{Imaginäre Zahlen}
\textbf{Konjugierte}

\textbf{Multiplikation}

\textbf{Division}


    

\subsubsection{Koordinaten Arten}
\begin{flalign}
    &\textbf{Kartesische Koordinaten}& \notag \\
    &z = x + iy &\\[1ex]
    &\textbf{Polarkoordinaten} & \notag \\
    &\text{Umrechnung kartesisch} \to \text{polar:} & \notag \\
    &r = \sqrt{x^2 + y^2} & \notag \\
    &\varphi = \arctan\left(\frac{y}{x}\right) \quad (x \neq 0) & \notag \\[1ex]
    &\textbf{Komplexe Polarform} & \notag \\
    &\operatorname{cis} \varphi = \cos \varphi + i \sin \varphi & \notag \\
    &z = r \operatorname{cis} \varphi = r(\cos \varphi + i \sin \varphi) & \label{eq:polarform} \\[1ex]
    &\textbf{Euler-Form} & \notag \\
    &z = re^{i\varphi} \quad \text{(äquivalent zur Polarform)} & \label{eq:euler}
\end{flalign}

\begin{flalign}
    &x = r \cdot \cos{\phi} \quad|\quad y = r \cdot \sin{\phi}&\\
    &\varphi = \arctan{\frac{x}{y}}&\\
    &z = re^{i\varphi} = x + iy = r\cdot \operatorname{cis}{\varphi}&
\end{flalign}
    
\[
\arg(z) =
\begin{cases} 
    0                                                & \text{if } \Re(z) \geq 0 \land \Im(z) = 0 \quad | \quad \text{CASE 1}\\
    \arctan\left(\frac{\Im(z)}{\Re(z)}\right)        & \text{if } \Re(z) > 0 \land \Im(z) > 0 \quad | \quad \text{CASE 2}\\
    \frac{\pi}{2}                                    & \text{if } \Re(z) = 0 \land \Im(z) > 0 \quad | \quad \text{CASE 3}\\
    \arctan\left(\frac{\Im(z)}{\Re(z)}\right) + \pi  & \text{if } \Re(z) < 0 \qquad \qquad \qquad \quad| \quad \text{CASE 4}\\
    \pi                                              & \text{if } \Re(z) < 0 \land \Im(z) = 0 \quad | \quad \text{CASE 5}\\
    \frac{3\pi}{2}                                   & \text{if } \Re(z) = 0 \land \Im(z) < 0 \quad | \quad \text{CASE 6}\\
    \arctan\left(\frac{\Im(z)}{\Re(z)}\right) + 2\pi & \text{if } \Re(z) > 0 \land \Im(z) < 0 \quad | \quad \text{CASE 7}
\end{cases}
\]

\begin{tikzpicture}[scale=1]

% Axes
\draw[->] (-2,0) -- (2,0) node[right] {$\Re(z)$};
\draw[->] (0,-2) -- (0,2) node[above] {$\Im(z)$};

% Regions
% Case 1: Re(z) >= 0 and Im(z) = 0 (positive real axis)
\draw[thick, red] (0,0) -- (2,0) node[midway, below] {Case 1};

% Case 2: Re(z) > 0 and Im(z) > 0 (upper-right quadrant)
\fill[green, opacity=0.3] (0,0) -- (0,2) -- (2,2) -- (2,0) -- cycle;
\node[green!50!black] at (1.5,1.5) {Case 2};

%  Case 3: Re(z) = 0 and Im(z) > 0 (positive imaginary axis)
\draw[thick, cyan] (0,0) -- (0,2) node[midway, left] {Case 3};

\fill[purple, opacity=0.3] (0,0) -- (0,2) -- (-2,2) -- (-2,-2) -- (0,-2) -- cycle;
% Case 4: Re(z) < 0 and Im(z) > 0 (upper-left quadrant)
\node[purple!70!black] at (-1.5,1.5) {Case 4};
% Case 4: Re(z) < 0 and Im(z) < 0 (lower-left quadrant)
\node[purple!70!black] at (-1.5,-1.5) {Case 4};

% Case 5: Re(z) < 0 and Im(z) = 0 (negative real axis)
\draw[thick, blue] (0,0) -- (-2,0) node[midway, below] {Case 5};

% Case 6: Re(z) = 0 and Im(z) < 0 (negativ imaginary axis)
\draw[thick, magenta] (0,0) -- (0,-2) node[midway, left] {Case 6};

% Case 7: Re(z) > 0 and Im(z) < 0 (lower-right quadrant)
\node[yellow!20!black] at (1.5,-1.5) {Case 7};
\fill[yellow, opacity=0.3] (0,0) -- (0,-2) -- (2,-2) -- (2,0) -- cycle;

% Origin
\filldraw[black] (0,0) circle (1pt) node[below right] {$0$};

\end{tikzpicture}
    \subsubsection{Koordinaten Wechsel}

\begin{flalign}
    &\textbf{Kartesisch $\Rightarrow$ Polar}&\notag\\
    &
    &\textbf{Polar $\Rightarrow$ Kartesisch}&\notag\\
    &&\\
\end{flalign}
    \section{Lineare Algebra}
    \subsection{Vektoranalysis}
    \subsubsection{Begriffe Vektoren}
\begin{itemize}
    \item \textbf{Skalarprodukt}\\
    $\bullet \langle \vec{v}, \vec{w} \rangle = 0 \Rightarrow \vec{v} \perp \vec{w} \qquad \bullet \langle a \cdot \vec{v}, \vec{w} \rangle = a \cdot \langle \vec{v}, \vec{w} \rangle$\\
    $\bullet \left(\begin{matrix} x \\ y \end{matrix}\right) \perp \left(\begin{matrix} -y\\ x \end{matrix} \right) \qquad \bullet \angle(\vec{v}, \vec{w}) = \arccos(\frac{\langle \vec{v}, \vec{w}}{\left|\vec{v}\right| \cdot \left|\vec{w}\right|})$
\end{itemize}


\subsubsection{Kreuzprodukt}
\begin{minipage}{\linewidth}
    \begin{tikzpicture}[yscale=1.5, xscale=1.7, every node/.style={scale=1}]
    % Rechteck
    \draw[-,fill=white!95!red](0,0)--(3,0)--(4,1)--(1,1)--cycle;
    % Formel in der Fläche
    \node at (2,0.5) {$A = |\textcolor{violet}{\vec{c}} | = |\textcolor{blue}{\vec{a}}| \cdot |\textcolor{red}{\vec{b}}| \cdot \sin(\varphi)$};
    % a
    \draw[ultra thick,-latex,blue](0,0)--(3,0)node[midway,below]{$\vec{a}$};
    % b
    \draw[ultra thick,-latex,red](0,0)--(1,1)node[midway,above]{$\vec{b}$};
    % a x b
    \draw[ultra thick,-latex,blue!50!red](0,0)--(0,2)node[pos=0.7,right]{$\textcolor{blue}{\vec{a}} \times \textcolor{red}{\vec{b}} = \vec{c}$};
    \draw (0.6,0) arc [start angle=0,end angle=45,radius=0.6]
    node[pos=0.7,right]{$\theta$};
    \end{tikzpicture}
\end{minipage}
    \subsubsection{Nabla Operator}
    \subsubsection{Tangentialebene}
\begin{flalign}
    &z = f(x_0; y_0) + \nabla f_x(x_0; y_0) \cdot (x - x_0) + \nabla f_y(x_0; y_0) \cdot (y - y_0)&\\
    &\vec{n} = \left(\begin{matrix}
        f_x(x_0; y_0)\\
        f_y(x_0; y_0)\\
        -z\\
    \end{matrix}\right)&\label{eq:Normalenvektor}
\end{flalign}
\vspace{4mm}
\ref{eq:Normalenvektor} Normalenvektor, welcher senkrecht auf der Tangentialebene steht\\

\begin{minipage}{0.4\linewidth}
    \textbf{Priorisierung um den Gradienten in die Tangentialebenenform zu bekommen}
    \begin{enumerate}
        \item Faktorisieren
        \item Additionsverfahren
        \item Umstellen und Einsetzen
    \end{enumerate}
\end{minipage}
\hfill
\begin{minipage}{0.4\linewidth}
    \textbf{Beispiel}\\
    \begin{flalign}
        &f(x,y) = x^3 + x^2 \cdot \ln{y^2 + 1} - 3x&\notag\\
        &\nabla f(x,y) = \begin{matrix}
            3x^2 - 2x \cdot \ln{y^2 + 1} - 3\\
            - \frac{2x^2y}{y^2 + 1}
        \end{matrix}&\notag\\
        &f(3;1) = 27 - 9\cdot \ln{2} -9 = 18 - 9\cdot \ln{2}&\notag\\
        &\nabla f_x (3; 1) = 27 - 6\ln{2} - 3 = 24 - 6\ln{2}&\notag\\
        &\nabla f_y(3;1) = -9&\notag\\
        &-45 + 9\ln{2} - 6x\ln{2} + 24x - 9y&\notag
    \end{flalign}
\end{minipage}
    \subsubsection{Totales Differential}

    \subsubsection{Hessematrix}
    \subsubsection{Extremwertstellen/Kritische Stellen}
\textcolor{red}{Lagrange Verfahren einfügen}
\begin{flalign}
    &\text{Kritische Stellen } = \nabla f \stackrel{!}{=} 0&\\
    &L(\vec{x}, \lambda) = f(\vec{x}) + \lambda \cdot g(\vec{x})&\\
    &\nabla L \stackrel{!}{=} 0&\\
\end{flalign}
    \subsection{Matrizen}
    \subsubsection{Standardmatrizen}
\vspace{3mm}
\begin{minipage}{0.4\linewidth}
    \begin{flalign}
        &\mathds{1} = \left(\begin{matrix}
            1 & 0\\
            0 & 1\\
        \end{matrix}\right)&\label{eq:Einheitsmatrix}\\
        &\mathbb{P} = \left(\begin{matrix}
            -1 & 0\\
            0 & -1\\
        \end{matrix}\right)&\label{eq:Punktspiegelung}\\
        &\mathbb{P_y}_x = \left(\begin{matrix}
            1 & 0\\
            0 & 0\\
        \end{matrix}\right)&\label{eq:Projektionsmatrix_X}\\
        &\mathbb{P_x}_y = \left(\begin{matrix}
            0 & 0\\
            0 & 1\\
        \end{matrix}\right)&\label{eq:Projektionsmatrix_Y}\\
        &\mathbb{Z}_a = \left(\begin{matrix}
            a & 0\\
            0 & a\\
        \end{matrix}\right)&\label{eq:Zentrische_Komponenten_Streckungs_Matrix}\\
        &A = \left(\begin{matrix}
            a & 0\\
            0 & b\\
        \end{matrix}\right)&\label{eq:Komponenten_Streckungs_Matrix}
    \end{flalign}
\end{minipage}
\hfill
\begin{minipage}{0.5\linewidth}
    \begin{flalign}
        &\mathbb{S_y}_x = \left(\begin{matrix}
            1 & 0\\
            0 & -1\\
        \end{matrix}\right)&\label{eq:X_Spiegelungs_Matrix}\\
        &\mathbb{S_x}_y = \left(\begin{matrix}
            -1 & 0\\
            0 & 1\\
        \end{matrix}\right)&\label{eq:Y_Spiegelungs_Matrix}\\
        &\mathbb{R}_{\frac{\pi}{2}} = \left(\begin{matrix}
            0 & -1\\
            1 & 0\\
        \end{matrix}\right)&\label{eq:180_Rotations_Matrix}\\
        &\mathbb{R_{\varphi}} = \left(\begin{matrix}
            \cos{\varphi} & -\sin{\varphi}\\
            \sin{\varphi} & \cos{\varphi}
        \end{matrix}\right)&\label{eq:Phi_Rotations_Matrix}\\
        &\mathbb{R_{-\varphi}} = \left(\begin{matrix}
            \cos{\varphi} & \sin{\varphi}\\
            -\sin{\varphi} & \cos{\varphi}
        \end{matrix}\right)&\label{eq:eq:Anti_Phi_Rotations_Matrix}
    \end{flalign}
\end{minipage}
\begin{multicols}{2}
    \begin{itemize}
        \item \ref{eq:Einheitsmatrix}: Einheitsmatrix
        \item \ref{eq:Punktspiegelung}: Punktspiegelungs-matrix
        \item \ref{eq:Projektionsmatrix_X}: Projektionsmatrix auf X-Achse
        \item \ref{eq:Projektionsmatrix_Y}: Projektionsmatrix auf Y-Achse
        \item \ref{eq:Zentrische_Komponenten_Streckungs_Matrix}: Zentrische Komponenten Streckungs Matrix
        \item \ref{eq:Komponenten_Streckungs_Matrix}: Komponenten Streckungs Matrix
        \item \ref{eq:X_Spiegelungs_Matrix} Spiegelungs-Matrix an der X-Achse
        \item \ref{eq:Y_Spiegelungs_Matrix} Spiegelungs-Matrix an der Y-Achse
        \item \ref{eq:180_Rotations_Matrix} Rotations-Matrix um den Ursprung 180\degree
        \item \ref{eq:Phi_Rotations_Matrix} Rotations-Matrix mit Winkel \varphi
        \item \ref{eq:eq:Anti_Phi_Rotations_Matrix} Rotations-Matrix mit Winkel \varphi \,im Gegenuhrezeigersin
    \end{itemize}
\end{multicols}
    \subsubsection{Determinante}
\textbf{2x2 Matrizen}\\
\begin{minipage}{0.29\linewidth}
    \vspace{3mm}
    \begin{flalign}
        & \det(
            \begin{matrix}
                $\tikzmarknode{a}{a}$ & $\tikzmarknode{b}{b}$\\
                $\tikzmarknode{c}{c}$ & $\tikzmarknode{d}{d}$\\
            \end{matrix}
        ) = &\notag
    \end{flalign}
    \begin{tikzpicture}[overlay, remember picture]
        \draw[-, red, thick] (a) to (d);
        \draw[-, blue, thick] (c) to (b);
    \end{tikzpicture}
\end{minipage}
\hfill
\begin{minipage}{0.39\linewidth}
    \begin{flalign}
        &\textcolor{red}{(a \cdot d)} - \textcolor{blue}{(c \cdot b)}&
    \end{flalign}
\end{minipage}\\
\textbf{3x3 Matrizen}\\
\begin{minipage}{0.39\linewidth}
    \vspace{3mm}
    \begin{flalign}
        & \det(
            \begin{matrix}
                $\tikzmarknode{a}{a}$ & $\tikzmarknode{b}{b}$ & $\tikzmarknode{c}{c}$\\
                $\tikzmarknode{d}{d}$ & $\tikzmarknode{e}{e}$ & $\tikzmarknode{f}{f}$\\
                $\tikzmarknode{g}{g}$ & $\tikzmarknode{h}{h}$ & $\tikzmarknode{i}{i}$\\
            \end{matrix}
        ) = &\notag
    \end{flalign}
    \begin{tikzpicture}[overlay, remember picture]
        \draw[-, red, thick] (a) to (e) to (i);
        \draw[-, red, thick] (b) to (f) to (g);
        \draw[-, red, thick] (c) to (d) to (h);
        \draw[-, blue, thick] (g) to (e) to (c);
        \draw[-, blue, thick] (h) to (f) to (a);
        \draw[-, blue, thick] (i) to (d) to (b);
    \end{tikzpicture}
\end{minipage}
\hfill
\begin{minipage}{0.59\linewidth}
    \begin{flalign}
        &\textcolor{red}{(a \cdot e \cdot i) + (b \cdot f \cdot g) + (c \cdot d \cdot h)}&\notag\\
        &\textcolor{blue}{- (g \cdot e \cdot c) - (h \cdot f \cdot a) - (i \cdot d \cdot b)}&
    \end{flalign}
\end{minipage}

% Define signed matrix colors
\definecolor{pos-sign}{RGB}{255,0,0}  % Red for '+'
\definecolor{neg-sign}{RGB}{0,0,255}  % Blue for '-'

% Custom command for colored sign entries
\newcommand{\sign}[1]{\if#1+{\color{pos-sign}+}\else{\color{neg-sign}-}\fi}

\textbf{4x4 Matrizen / nxn Matrix}\\
\begin{minipage}{0.3\linewidth}
    \vspace{3mm}
    \begin{flalign}
    &\mathrel{\vcenter{\hbox{
        \begin{tikzpicture}[baseline=(A.base), remember picture]
        % Original coefficient matrix
        \matrix [matrix of nodes, nodes={inner sep=2mm}, anchor=base, left delimiter=(, right delimiter=),
                ampersand replacement=\&] (A) {
            \tikzmarknode{a}{a} \& \tikzmarknode{b}{b} \& \tikzmarknode{c}{c} \& \tikzmarknode{d}{d}\\
            \tikzmarknode{e}{e} \& \tikzmarknode{f}{f} \& \tikzmarknode{g}{g} \& \tikzmarknode{h}{h}\\
            \tikzmarknode{i}{i} \& \tikzmarknode{j}{j} \& \tikzmarknode{k}{k} \& \tikzmarknode{l}{l}\\
            \tikzmarknode{m}{m} \& \tikzmarknode{n}{n} \& \tikzmarknode{o}{o} \& \tikzmarknode{p}{p}\\
        };
        % Overlay sign matrix (transparent)
        \matrix [matrix of nodes, opacity=0.7,
                nodes={inner sep=2mm}, overlay, ampersand replacement=\&,
                shift={(0.2,0.15)}] at (A) {
            \sign{+} \& \sign{-} \& \sign{+} \& \sign{-}\\ 
            \sign{-} \& \sign{+} \& \sign{-} \& \sign{+}\\ 
            \sign{+} \& \sign{-} \& \sign{+} \& \sign{-}\\ 
            \sign{-} \& \sign{+} \& \sign{-} \& \sign{+}\\ 
        };
    \end{tikzpicture}}}} = A&\notag
    \end{flalign}
\end{minipage}
\hfill
\begin{minipage}{0.5\linewidth}
    \begin{enumerate}
        \item Spalte/Reihe mit den meisten Nullen auswählen
        \item Die Vorfaktoren der Spalten/Reihenwerte mit der Vorzeichenmatrix (\textcolor{red}{+}, \textcolor{blue}{-}) bestimmen
    \end{enumerate}
\end{minipage}

\begin{minipage}{0.45\linewidth}
    test
\end{minipage}
\hfill
\begin{minipage}{0.48\linewidth}
    \begin{enumerate}[start=3]
        \item test
    \end{enumerate}
\end{minipage}
    \subsubsection{Inverse berechnen mittels Adjunkter Matrix}
\begin{minipage}{0.5\linewidth}
    \textbf{Benötigt}\\
    \begin{itemize}
        \item Det(A) $\ne 0$
        \item 2x2 im Kopf | 3x3 mit Taschenrechner | 3x3 - nxn mit Adjunkte Matrix
    \end{itemize}
\end{minipage}
\hfill
\begin{minipage}{0.5\linewidth}
    \textbf{Wichtiges}\\
    \begin{itemize}
        \item Det(A) = 0 $\nRightarrow A^{-1}$\\ Det(A) $\ne 0 \Rightarrow A^{-1}$
        \item $(a \cdot A)^{-1} = \frac1a \cdot A^{-1}$
        \item $(A \cdot B)^{-1} = B^{-1} \cdot A^{-1}$
        \item $(A^T)^{-1} = (A^{-1})^T$
        \item $(A^{-1})^{-1} = A$
    \end{itemize}
\end{minipage}

\textbf{2x2 Matrix}\\
\begin{flalign}
    &A = \left[\begin{matrix}
        a & b\\
        c & d
    \end{matrix}\right] \qquad A^{-1} = \frac{1}{ad - bc} \cdot \left[\begin{matrix}
        d & -b\\
        -c & a
    \end{matrix}\right]&
\end{flalign}

\textbf{Anleitung 3x3, 4x4 - nxn}\\
\begin{flalign}
    &A^{-1} = \frac{1}{\det(A)} \cdot \scalebox{0.55}{$\displaystyle \left[
        \begin{matrix}
            {\color{red}\boxed{+}}
            \det(
            \begin{matrix}
                f & g & h\\
                j & k & l\\
                n & o & p
            \end{matrix}) &
            {\color{blue}\boxed{-}}
            \det(
            \begin{matrix}
                e & g & k\\
                i & k & l\\
                m & o & p
            \end{matrix}) &
            {\color{red}\boxed{+}}
            \det(
            \begin{matrix}
                e & f & h\\
                i & j & l\\
                m & n & p  
            \end{matrix}) &
            {\color{blue}\boxed{-}}
            \det(
            \begin{matrix}
                e & f & g\\
                i & j & k\\
                m & n & o
            \end{matrix})
            \\
            {\color{blue}\boxed{-}}
            \det(
            \begin{matrix}
                b & c & d\\
                j & k & l\\
                n & o & p
            \end{matrix}) &
            {\color{red}\boxed{+}}
            \det(
            \begin{matrix}
                a & c & d\\
                i & k & l\\
                m & o & p
            \end{matrix}) &
            {\color{blue}\boxed{-}}
            \det(
            \begin{matrix}
                a & b & d\\
                i & j & l\\
                m & n & p  
            \end{matrix}) &
            {\color{red}\boxed{+}}
            \det(
            \begin{matrix}
                a & b & c\\
                i & j & k\\
                m & n & o
            \end{matrix})
            \\
            {\color{red}\boxed{+}}
            \det(
            \begin{matrix}
                b & c & d\\
                f & g & h\\
                n & o & p
            \end{matrix}) &
            {\color{blue}\boxed{-}}
            \det(
            \begin{matrix}
                a & c & d\\
                e & g & h\\
                m & o & p
            \end{matrix}) &
            {\color{red}\boxed{+}}
            \det(
            \begin{matrix}
                a & b & d\\
                e & f & h\\
                m & n & p  
            \end{matrix}) &
            {\color{blue}\boxed{-}}
            \det(
            \begin{matrix}
                a & b & c\\
                e & f & g\\
                m & n & o
            \end{matrix})
            \\
            {\color{blue}\boxed{-}}
            \det(
            \begin{matrix}
                b & c & d\\
                f & g & h\\
                j & k & l
            \end{matrix}) &
            {\color{red}\boxed{+}}
            \det(
            \begin{matrix}
                a & c & d\\
                e & g & h\\
                j & k & l
            \end{matrix}) &
            {\color{blue}\boxed{-}}
            \det(
            \begin{matrix}
                a & b & d\\
                e & f & h\\
                i & j & l  
            \end{matrix}) &
            {\color{red}\boxed{+}}
            \det(
            \begin{matrix}
                a & b & c\\
                e & f & g\\
                i & j & k
            \end{matrix})
    \end{matrix}\right]^{\mathbf{T}}
    $}&\label{eq:4x4_Inverse_berechnen}
\end{flalign}

\begin{enumerate}
    \item Determinante  von A berechnen
    \item Determinanten der inneren Matrizen berechnen und das Ergebnis mit jeweiligem Vorzeichen eintragen (3x3 Spaltenweise mit TS ausrechnen)
    \item Matrix transponieren
    \item $A \cdot A^{-1} = \mathbb{I}$ prüfen
\end{enumerate}
    \subsubsection{Eigenwerte}
    \subsubsection{Spur}
Alle Einträge der Matrix auf der Diagonalen summiert.
Aussagefähigkeit:
\begin{itemize}
    \item $trace(A) = \sum_{0}^{n}{\lambda_{n}}$
\end{itemize}

}
\end{document}