\subsubsection{Parameteriesierte Kurve}
\vspace{3mm}
\begin{minipage}{0.4\linewidth}
    \begin{flalign}
        &s: \left[\tau_{0}, \tau_{E}\right] \Rightarrow \mathbb{R}^n&\notag
    \end{flalign}    
\end{minipage}
\hfill
\begin{minipage}{0.5\linewidth}
    \begin{flalign}
        &\tau \mapsto \vec{s}(\tau) := \left[\begin{matrix}
            s_1(\tau)\\
            s_2(\tau)\\
            \vdots\\
            s_n(\tau)\\
        \end{matrix}\right]&
    \end{flalign}
\end{minipage}
\begin{itemize}
    \item Geschwindigkeitsvektor: $\vec{v}(\tau) := \dot{s}(\tau)$
    \item Bahngeschwindigkeit: $\vec{v}(\tau) := \left|\vec{v}(\tau)\right|$
    \item Bahnvektor für $\vec{v}(\tau) \ne 0: \hat{e}(\tau) := \hat{v}(\tau)$
    \item Beschleunigungsvektor: $\vec{a}(\tau) := \dot{v}(\tau)$
    \item Bahnbeschleunigung: $a_B(\tau) := \langle a(\tau), \hat{e}(\tau)⟩$
    \item Bahn: $B := \vec{s}(\left[\tau_0, \tau_E\right])$
    \item Ortsvektor zeigt von Ursprung auf Punkt der Bahn
    \end{itemize}

\subsubsection{Standardkurven}
\begin{minipage}{0.45\linewidth}
    \begin{flalign}
        &\textbf{Kreis mit Mittelpunkt M}&\notag\\
        &s(\tau) = M + \left[\begin{matrix}
            R \cdot \cos{\tau}\\
            R \cdot \sin{\tau}
        \end{matrix}\right]&\\
        &\textbf{Zylinder}&\notag\\
        &P(\varphi;z) = \left[\begin{matrix}
            R \cdot \cos{\varphi}\\
            R \cdot \sin{\varphi}\\
            z\\
        \end{matrix}\right]&\\
        &\varphi \in \left[ 0, 2\pi \right[ ;z \in \left[0, H\right]&\notag\\
        &\textbf{Kugel}&\notag\\
            &P(\theta;\varphi) = \left[\begin{matrix}
            R \cdot \sin{\theta} \cdot \cos{\varphi}\\
            R \cdot \sin{\theta} \cdot \sin{\varphi}\\
            R \cdot \cos{\theta}\\
        \end{matrix}\right]&\\
        &\theta \in \left[ 0, \pi \right[ ;\varphi \in \left[0, 2\pi\right[&\notag
    \end{flalign}
\end{minipage}
\hfill
\begin{minipage}{0.5\linewidth}
    \begin{flalign}
        &\textbf{Kegel}&\notag\\
        &P(\varphi;r) = \left[\begin{matrix}
            r \cdot \cos{\varphi}\\
            r \cdot \sin{\varphi}\\
            \frac{H}{R} \cdot r\\
        \end{matrix}\right]&\\
        &\varphi \in \left[ 0, 2\pi \right[ ;r \in \left[0, R\right]&\notag\\
        &\textbf{Turnus}&\notag\\
            &P(\theta;\varphi) = \left[\begin{matrix}
            (R + r \cdot \sin{\theta}) \cdot \cos{\varphi}\\
            (R + r \cdot \sin{\theta}) \cdot \sin{\varphi}\\
            r \cdot \cos{\theta}\\
        \end{matrix}\right]&\\
        &\theta , \varphi \in \left[0, 2\pi\right[&\notag
    \end{flalign}
\end{minipage}

R = Radius | r = Variable

\subsubsection{Bogenlänge}
\begin{flalign}
    &\Delta s := \int_{\tau_0}^{\tau_E}{v(\tau)} \,d\tau&
\end{flalign}

\subsubsection{Linienintegral}
\begin{flalign}
    &I := \int_{\tau_0}^{\tau_E}{\langle \vec{w}, \vec{v} \rangle} \,d\tau = \int_{s_0}^{s_E}{\langle \vec{w}, \hat{e} \rangle} \,ds&
\end{flalign}
\begin{itemize}
    \item Vektorfeld $\vec{w}$
    \item Geschwindigkeitsvektor Parameteriesierte Kurve: $\vec{v}$
    \item Sei $\langle \vec{w}, \hat{e} \rangle =: C \equiv$ konst dann gilt: $I = C \cdot \Delta s$
\end{itemize}