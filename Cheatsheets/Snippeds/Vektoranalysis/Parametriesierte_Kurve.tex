\subsubsection{Parameteriesierte Kurve}
\vspace{3mm}
\begin{minipage}{0.4\linewidth}
    \textbf{Anleitung}
    \begin{enumerate}
        \item Funktion zeichnen
        \item $\tau$ einsetzen und Wert für jeweilige Achse bestimmen
        \item Grenzen bestimmen
    \end{enumerate}
\end{minipage}
\hfill
\begin{minipage}{0.5\linewidth}
    \textbf{Formeln}\\
    \begin{flalign}
        &\gamma: \left[\tau_{0}, \tau_{E}\right] \Rightarrow \mathbb{R}^n&\notag\\
        &\tau \mapsto \vec{\gamma}(\tau) := \left[\begin{matrix}
            \gamma_1(\tau)\\
            \gamma_2(\tau)\\
            \vdots\\
            \gamma_n(\tau)\\
        \end{matrix}\right]&
    \end{flalign}
\end{minipage}

\begin{minipage}{0.5\linewidth}
    \begin{itemize}
        \item Geschwindigkeitsvektor:\\ $\vec{v}(\tau) := \dot{\vec{\gamma}}(\tau)$
        \item Bahngeschwindigkeit:\\ $\vec{v}(\tau) := \left|\vec{v}(\tau)\right|$
        \item Bahnvektor für $\vec{v}(\tau) \ne 0:\\ \hat{e}(\tau) := \hat{v}(\tau)$
        \item Beschleunigungsvektor:\\ $\vec{a}(\tau) := \dot{v}(\tau)$
    \end{itemize}
\end{minipage}
\begin{minipage}{0.5\linewidth}
    \begin{itemize}
        \item Bahnbeschleunigung:\\ $a_B(\tau) := \langle a(\tau), \hat{e}(\tau)⟩$
        \item Bahn:\\ $B := \vec{\gamma}(\left[\tau_0, \tau_E\right])$
        \item Ortsvektor zeigt von Ursprung auf Punkt der Bahn
        \item $x^2 \Rightarrow \vec{\gamma} = \left(\begin{matrix}
            t\\
            t^2
        \end{matrix}\right)$
    \end{itemize}
\end{minipage}