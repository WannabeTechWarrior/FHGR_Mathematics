\subsubsection{Richtungsableitung}
\begin{minipage}{0.6\linewidth}
    \textbf{Formel}\\
    \begin{flalign}
        &\frac{\nabla f(x_0; y_0)^T \cdot \vec{r}}{\left|\vec{r}\right|} \label{eq:Richtungsableitung}&\\
        &\text{Länge des steilsten Anstiegs } = \left|\nabla f\right| \label{eq:Steilster_Anstieg}&\\
        &\text{Steigungswinkel } = \arctan{\left| \nabla f \right|} \label{eq:Steigungswinkel}&
    \end{flalign}\\
    Folgendes muss gegeben sein:
    \begin{enumerate}
        \item Richtung $\vec{r}$
        \item Punkt $P(x_0;y_0)$
        \item Gradient $\nabla f$
    \end{enumerate}
\end{minipage}
\hfill
\begin{minipage}{0.4\linewidth}
    \textbf{Anleitung}
    \begin{enumerate}
        \item Gradient berechnen
        \item Betrag des Richtungsvektor berechnen
        \item Punkt in Gradient einsetzen
        \item Mit Formel \ref{eq:Richtungsableitung} berechnen
        \item Steilster Ansteig berechnen \ref{eq:Steilster_Anstieg}
        \item Steigungswinkel \ref{eq:Steigungswinkel}
    \end{enumerate}
\end{minipage}