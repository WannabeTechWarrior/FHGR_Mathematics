% Create variables
\newcommand{\varA}{\ensuremath{\textcolor{brown}{\frac{1}{2}}}} 
\newcommand{\varB}{\ensuremath{\textcolor{magenta}{\frac{3}{2}}}}
\newcommand{\TeilerPol}{{\color{blue}q(x)} }
\newcommand{\SolPol}{{\color{LimeGreen}S(x)} }
\newcommand{\RestPol}{{\color{purple}r(x)} }
\newcommand{\Pol}{{\color{teal}P(x)} }
\newcommand{\M}[2]{\tikzmarknode{#1}{#2}}

\subsubsection{Hornerschema}
\begin{minipage}{0.45\linewidth}
    \textbf{Benötigt/Wichtig}\\
    \begin{itemize}
        \item Leitkoeffizient von \TeilerPol = 1
        \item Erratene Nullstelle
        \item grad(r) < grad(q)
    \end{itemize}
    \textbf{Formel}\\
    \begin{flalign}
        &\frac{\Pol}{\TeilerPol} = \SolPol + \frac{\RestPol}{\TeilerPol}& \label{eq:Hornerschema}
    \end{flalign}
\end{minipage}
\hfill
\begin{minipage}{0.55\linewidth}
    \textbf{Möglichkeiten}
    \begin{itemize}
        \item Polynom an einer Steller auswerten
        \item Polynomdivision
        \item Zahlensystem umformen
        \item Taylor Reihen entwickeln \textcolor{red}{noch hinzufügen} % Mathepeter video: Polynome nach Potenzen entwickeln (mit Horner Schema)
    \end{itemize}
\end{minipage}

\textbf{Beispiel}\\
$P(x) = -2x^5 - 3x^4 - 8x^3 - 11x^2 + 4$\\

$q(x) = -2x^2 + 3x -1 = \boxed{\mathbf{\textcolor{cyan}{1}}} -2 (x^2 - \varB + \varA)$\\

\begin{tabular}{>{$}r<{$} | *{6}{>{$}r<{$} }}
    \boxed{\textcolor{cyan}{2}}  & \M{01}{-2} & \M{02}{-3} & \M{03}{-8} & \M{04}{-11} & \M{05}{0} & \M{06}{4}\\ 
        \hline
    \M{10}{\varA} & \M{11}{\cdot} & \M{12}{\cdot} & \M{13}{1} & \M{14}{3} & \M{15}{8} & \M{16}{16}\\
    \M{20}{\varB} & \M{21}{\cdot} & \M{22}{-3} & \M{23}{-9} & \M{24}{-24} & \M{25}{-48} & \cdot \\
        \hline
    \boxed{\textcolor{cyan}{4}}   & \multicolumn{4}{c}{{\color{LimeGreen}\boxed{\M{31}{-2} \quad \M{32}{-6} \quad \M{33}{-16} \quad \M{34}{-32}}}} & \multicolumn{2}{c}{{\color{purple}\boxed{\M{35}{-40} \quad \M{36}{20}}}} 
\end{tabular}

\begin{flalign*}
    &\frac{\Pol}{\TeilerPol} \Rightarrow \overbrace{-\frac12}^{\boxed{\mathbf{\textcolor{cyan}{5}}}} \left({\color{LimeGreen}\boxed{-2x^3 -6x^2 - 16x - 32}} + \frac{{\color{purple}\boxed{-40x + 20}}}{{\color{blue}\boxed{x^2 - \frac32 x + \frac12}}}\right)&\\
    &= x^3 + 3x^2 + 8x + 16 + \frac{-40x + 20}{-2x^2 + 3x -1}&
\end{flalign*}

\textbf{Anleitung}
\begin{enumerate}[label={\color{cyan}\arabic*.}]
    \item Leitkoeffizient von \TeilerPol auf 1 setzen
    \item Tabelle mit den Koeffizienten und den Punkten erstellen
    \item Den Pfeilen nach unten entlang \textcolor{red}{addieren} und nach oben  mit jeweiligem Zeilenfaktor (links) \textcolor{blue}{multiplizieren}
    \item Gr(\Pol) - Gr(\TeilerPol) = Grad(\SolPol). Der Rest gehört zu \RestPol
    \item Wenn Schritt 1 dann muss durch den ausgeklammerten Faktor geteilt werden. Bei \RestPol Faktor in den Nenner ziehen.
\end{enumerate}

\begin{tikzpicture}[overlay, remember picture]
    \draw[->, red, thin] (01) to node[red, left] {+} (31);
    \draw[->, blue, thin] (31) to node[sloped] {\cdot \varB} (22);
    \draw[->, blue, thin] (31) to [out=5, in=220] node[pos=0.8, above, sloped] {\cdot \varA} (13);
    \draw[->, red, thin] (02) to (32);
    \draw[->, blue, thin] (32) to (14);
    \draw[->, red, thin] (03) to [bend left] (33);
    \draw[->, blue, thin] (33) to (15);
    \draw[->, red, thin] (04) to [bend left] (34);
    \draw[->, blue, thin] (34) to (16);
    \draw[->, red, thin] (05) to [bend left] (35);
    \draw[->, red, thin] (06) to [bend left] (36);
\end{tikzpicture}

% Erase variables
\makeatletter
    \let\varA\@undefined 
    \let\varB\@undefined 
    \let\TeilerPol\@undefined
    \let\Pol\@undefined
    \let\RestPol\@undefined
    \let\SolPol\@undefined
    \let\M\@undefined
\makeatother