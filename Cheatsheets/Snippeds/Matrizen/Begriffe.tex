\subsubsection{Begriffe}
\begin{itemize}
    \item \textbf{Rechenregel}\\
    \textbf{Transposition}\\ 
    $\bullet  (A^T)^T = a \qquad \bullet (A + B)^T = A^T + B^T \qquad \bullet (a \cdot A)^T = a \cdot A^T$\\
    \textbf{Multiplikation}\\
    $\bullet A^{n_1 \times m} \cdot B^{m \times n_2} \qquad \bullet (a \cdot A) \cdot B = a \cdot (B \cdot A) = A \cdot (a \cdot B)$\\
    $\bullet (A \cdot B) \cdot C = A \cdot (B \cdot C) \qquad \bullet A \cdot (B \cdot C) = A \cdot B + A \cdot C$\\
    $\bullet (A + B) \cdot C = A \cdot C + B \cdot C \qquad \bullet (A \cdot B)^T = B^T \cdot A^T$
    \item \textbf{Spektrum}\\
    Menge der Eigenvektoren
    \item \textbf{Spur}\\ 
    Diagonale addiert $\tr(A) = A_1^1 + A_1^1 + \cdots + A_n^n = \sum_{0}^{n}{\lambda_{n}}$\\
    $\bullet \tr(A^T) = \tr(A) \qquad \qquad \bullet \tr(A + B) = \tr(A) + \tr(B)$\\
    $\bullet \tr(a \cdot A) = a \cdot \tr(A) \qquad \bullet \tr(A \cdot B) = \tr(B \cdot A)$
    \item \textbf{Bild und Kern}\\ 
    \textbf{Bild}\\
    img$(a) = a(V) = \{\vec{w} \in \mathbb{W} \,|\, \vec{v} \in \mathbb{V}$ mit $a(\vec{v}) = \vec{w}\}$\\
    \textbf{Kern}\\
    $\ker(a) = \{\vec{v} \in \mathbb{V} \,|\, a(\vec{v}) = 0\}$\\
    $A \cdot \text{Kern} = \vec{0}$\\
    Ist $\ker(a) = 0$ dann hat a einen trivialen Kern. Eine reguläre Matrix hat einen trivialen Kern.\\
    \textbf{Dimensionssatz}\\
    $\text{dim(kern(A))} + \text{dim(img(A))} = \text{dim(A)}$\\
    \item \textbf{Regulärsatz}\\ 
    Einer quadratische Matrix A ist genau dann regulär, wenn gilt $\det(A) \ne 0$
    \item \textbf{Orthogonale Matrizen}\\
    $A^{-1} = A^T$\\
    $A^T \cdot A = \mathds{1}$
    \item \textbf{Quadratische Matrizen}\\
    $\bullet n^2$ Komponenten\\
    $\bullet A^3 = A \cdot A \cdot A$
w
\end{itemize}