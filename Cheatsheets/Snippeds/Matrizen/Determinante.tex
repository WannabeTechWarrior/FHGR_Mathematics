\subsubsection{Determinante}
\textbf{Regeln}
\begin{itemize}
    \item $\det(A^{T}) = \det(A)$
    \item $\det(a \cdot A^{T}) = a^n \cdot \det(A)$
    \item $\det(A \cdot B) = \det(A) \cdot \det(B)$
    \item $\det(A^{-1}) = \frac{1}{\det(A)}$ falls A regulär
    \item Zeilen-/Spaltentausch\\ $\det(A) \mapsto -\det(A)$
    \item Multiplikation einer Zeile/Spalte mit a\\ $\det(A) \mapsto a \cdot \det(A)$
    \item Invarianz: Subtrahiert man von einer Zeile ein vielfaches einer anderen Zeile, so ändert sich die Determinante nicht
\end{itemize}

\textbf{2x2 Matrizen}\\
\begin{minipage}{0.29\linewidth}
    \vspace{3mm}
    \begin{flalign}
        & \det(
            \begin{matrix}
                $\tikzmarknode{a}{a}$ & $\tikzmarknode{b}{b}$\\
                $\tikzmarknode{c}{c}$ & $\tikzmarknode{d}{d}$\\
            \end{matrix}
        ) = &\notag
    \end{flalign}
    \begin{tikzpicture}[overlay, remember picture]
        \draw[-, red, thick] (a) to (d);
        \draw[-, blue, thick] (c) to (b);
    \end{tikzpicture}
\end{minipage}
\hfill
\begin{minipage}{0.39\linewidth}
    \begin{flalign}
        &\textcolor{red}{(a \cdot d)} - \textcolor{blue}{(c \cdot b)}& \label{eq:2x2_Determinante}
    \end{flalign}
\end{minipage}\\
\textbf{3x3 Matrizen}\\
\begin{minipage}{0.39\linewidth}
    \vspace{3mm}
    \begin{flalign}
        & \det(
            \begin{matrix}
                $\tikzmarknode{a}{a}$ & $\tikzmarknode{b}{b}$ & $\tikzmarknode{c}{c}$\\
                $\tikzmarknode{d}{d}$ & $\tikzmarknode{e}{e}$ & $\tikzmarknode{f}{f}$\\
                $\tikzmarknode{g}{g}$ & $\tikzmarknode{h}{h}$ & $\tikzmarknode{i}{i}$\\
            \end{matrix}
        ) = &\notag
    \end{flalign}
    \begin{tikzpicture}[overlay, remember picture]
        \draw[-, red, thick] (a) to (e) to (i);
        \draw[-, red, thick] (b) to (f) to (g);
        \draw[-, red, thick] (c) to (d) to (h);
        \draw[-, blue, thick] (g) to (e) to (c);
        \draw[-, blue, thick] (h) to (f) to (a);
        \draw[-, blue, thick] (i) to (d) to (b);
    \end{tikzpicture}
\end{minipage}
\hfill
\begin{minipage}{0.59\linewidth}
    \begin{flalign}
        &\textcolor{red}{(a \cdot e \cdot i) + (b \cdot f \cdot g) + (c \cdot d \cdot h)}&\notag\\
        &\textcolor{blue}{- (g \cdot e \cdot c) - (h \cdot f \cdot a) - (i \cdot d \cdot b)}& \label{eq:3x3_Determinante}
    \end{flalign}
\end{minipage}

% Define signed matrix colors
\definecolor{pos-sign}{RGB}{255,0,0}  % Red for '+'
\definecolor{neg-sign}{RGB}{0,0,255}  % Blue for '-'

% Custom command for colored sign entries
\newcommand{\sign}[1]{\if#1+{\color{pos-sign}+}\else{\color{neg-sign}-}\fi}

\textbf{4x4 Matrizen / nxn Matrix}\\
\begin{minipage}{0.3\linewidth}
    \vspace{3mm}
    \begin{flalign}
    &\mathrel{\vcenter{\hbox{
        \begin{tikzpicture}[baseline=(A.base), remember picture]
        % Original coefficient matrix
        \matrix [matrix of nodes, nodes={inner sep=2mm}, anchor=base, left delimiter=(, right delimiter=),
                ampersand replacement=\&] (A) {
            \tikzmarknode{a}{a} \& \tikzmarknode{b}{b} \& \tikzmarknode{c}{c} \& \tikzmarknode{d}{d}\\
            \tikzmarknode{e}{e} \& \tikzmarknode{f}{f} \& \tikzmarknode{g}{g} \& \tikzmarknode{h}{h}\\
            \tikzmarknode{i}{i} \& \tikzmarknode{j}{j} \& \tikzmarknode{k}{k} \& \tikzmarknode{l}{l}\\
            \tikzmarknode{m}{m} \& \tikzmarknode{n}{n} \& \tikzmarknode{o}{o} \& \tikzmarknode{p}{p}\\
        };
        % Overlay sign matrix (transparent)
        \matrix [matrix of nodes, opacity=0.7,
                nodes={inner sep=2mm}, overlay, ampersand replacement=\&,
                shift={(0.2,0.15)}] at (A) {
            \sign{+} \& \sign{-} \& \sign{+} \& \sign{-}\\ 
            \sign{-} \& \sign{+} \& \sign{-} \& \sign{+}\\ 
            \sign{+} \& \sign{-} \& \sign{+} \& \sign{-}\\ 
            \sign{-} \& \sign{+} \& \sign{-} \& \sign{+}\\ 
        };
    \end{tikzpicture}}}} = A^{4x4}&\notag
    \end{flalign}
\end{minipage}
\hfill\hspace{-7mm}
\begin{minipage}{0.6\linewidth}
    \begin{enumerate}
        \item Spalte/Reihe mit den meisten Nullen auswählen
        \item Die Vorfaktoren der Spalten/Reihenwerte mit der Vorzeichenmatrix (\textcolor{red}{+}, \textcolor{blue}{-}) bestimmen
    \end{enumerate}
\end{minipage}

\begin{equation}
    \det(A^{4\times4}) = \overbrace{D_a + D_e + D_i + D_m}^{\boxed{4}} =
        \begin{cases}
            \begin{aligned}
            D_a &= {\color{red}\boxed{+}} a \cdot \det\begin{pmatrix}
                        f & g & h\\ 
                        j & k & l\\
                        n & o & p
                    \end{pmatrix} \\ 
            D_e &= {\color{blue}\boxed{-}} e \cdot \det\begin{pmatrix}
                        b & c & d\\
                        j & k & l\\
                        n & o & p
                    \end{pmatrix} \\
            \boxed{3}\\
            D_i &= {\color{red}\boxed{+}} i \cdot \det\begin{pmatrix}
                        b & c & d\\
                        f & g & h\\ 
                        n & o & p
                    \end{pmatrix} \\
            D_m &= {\color{blue}\boxed{-}} m \cdot \det\begin{pmatrix}
                        b & c & d\\
                        f & g & h\\
                        j & k & l
                    \end{pmatrix}
            \end{aligned}
        \end{cases}
    \label{eq:nxn_Determinante}
\end{equation}
\begin{enumerate}[start=3]
    \item 3x3 Matrizen aufstellen durch abdecken der Zeilen und Spalten des jeweiligen Vorfaktors
    \item Ergebnisse addieren ergibt die Determinante der 4x4 Matrix
\end{enumerate}